The basic use case for HTBAC is to enable large scale free energy studies of 
protein-ligand binding using ensemble simulations.
We have already briefly introduced two free energy calculation 
protocols, ESMACS and TIES ~\cite{Wan2017brd4, Bhati2017}, with reference to the comparative rigour of the former relative to the later. 
Here we provide more details of their practical specifications and of the 
protein system we used to benchmark and refine them in this work.

\subsection{ESMACS and TIES}

In both protocols, ensembles of MD simulations are employed to perform averaging and to obtain tight control of error bounds in our estimates. 
In addition, the ability to run replica simulations concurrently means that, as long as sufficient compute resources are available, turn around times can be significantly reduced compared to the generation of single long trajectories.
Due to their shared philosophical underpinning both protocols share similar middleware requirements.

The current implementation of both TIES and ESMACS uses the the NAMD 
package~\cite{Phillips2005} to conduct the simulations.
Conceptually, each replica simulation consists of three stages: minimization, equilibration 
and production MD.
In practice the equilibration phase is broken into multiple steps to ensure that the size of the 
simulation box does not alter too much over the simulation.
During these steps positional constraints are gradually released from the structure and the 
system is heated to a physiologically realistic temperature.
Whilst both protocols share a common workflow for individual replicas, the make up of the 
ensemble is different.

In the case of ESMACS, an ensemble consists of a set of identical simulations differing only in
the initial velocities assigned to each atom.
Upon completion of the MD simulation, free energy computation (via MMPBSA and potentially normal 
mode analysis) is performed using AmberTools~\cite{amber14, Case2005, MillerIII2012}.

Whereas, all ESMACS replicas sample using the same system description, in TIES sub-ensembles are executed 
at diffent points along a transformation parameter, $\lambda$.
As $\lambda$ increases from zero to one the system description transforms from containing an 
initial drug to a target via a series of hybrid states.
The change in free energy associated with this transformation is calculated by integrating the 
values of the $dU/d\lambda$ across the full set of $\lambda$ windows simulated.
Obtaining accurate and precise results from TIES requires that the $\lambda$ windows correctly 
capture the changes of $dU/d\lambda$ over the transformation.
This behaviour may vary considerably between systems.
Typically, windows are evenly spaced between 0 and 1 with the spacing between them set before 
execution at a distance determined by the simulator to be sufficient for a wide raneg of systems. 
Typically, a TIES ensemble 65 replicas evenly distributed between 13 $\lambda$ windows
In order to obtain a meaningful TIES result it is necessary to not only simulate the drug pair 
in the protein but also in an aqueous environemnt, adding a further 65 replicas albeit it using a
smaller system at lower computational cost.

Following simulation (and in the case of ESMACS free energy analysis steps) both protocols require
the execution of short serial steps to provide summary statistics.
Both protocols can are highly customizable, for example the number of simulation replicas in the 
ensemble and the lengths of their runs can be varied.
More generally though, there may be cases where it is important to increase the sampling of phase 
space possibly through expanding the ensemble or in TIES changing the distributon of $\lambda$ 
windows.
In addition, the ESMACS protocol can also be extended to account for adaptation energies involved 
in altering the conformation of the protein or ligand during binding through the use of separate 
component simulations.

%High level description of MD step here%

%Describe generic TIES & ESMACS workflows%

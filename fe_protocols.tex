The basic use case for HTBAC is to enable large scale free energy studies of
protein-ligand binding using ensemble simulations.
We have already briefly introduced two free energy calculation
protocols, ESMACS and TIES ~\cite{Wan2017brd4, Bhati2017}, with reference to the comparative rigour of the former relative to the later.
Here we provide more details of their practical specifications and of the
protein system we used to benchmark and refine them in this work.

\subsection{ESMACS and TIES}

In both protocols, ensembles of MD simulations are employed to perform averaging and to obtain tight control of error bounds in our estimates.
In addition, the ability to run replica simulations concurrently means that, as long as sufficient compute resources are available, turn around times can be significantly reduced compared to the generation of single long trajectories.
Due to their shared philosophical underpinning both protocols share similar middleware requirements.

The current implementation of both TIES and ESMACS uses the the NAMD
package~\cite{Phillips2005} to conduct the simulations.
Conceptually, each replica simulation consists of three stages: minimization, equilibration
and production MD.
In practice the equilibration phase is broken into multiple steps to ensure that the size of the
simulation box does not alter too much over the simulation.
During these steps positional constraints are gradually released from the structure and the
system is heated to a physiologically realistic temperature.
Whilst both protocols share a common workflow for individual replicas, the make up of the ensemble is different.

In the case of ESMACS, an ensemble consists of a set of identical simulations differing only in
the initial velocities assigned to each atom.
Upon completion of the MD simulation, free energy computation (via MMPBSA and potentially normal
mode analysis) is performed using AmberTools~\cite{amber14, Case2005, MillerIII2012}.

TIES is a so called ``alchemical'' method in which the fact that the potential used to decribe the system is user definable to transform one
system into another.
This allows for the calculation of free energy differences between the two
systems, such as those induced by an alteration to a candidate drug.
Typically a transformation parameter (called $\lambda$) is defined such that as it increases from zero to one the system description transforms from containing an
initial drug to a target compound via a series of hybrid states.
Sampling along $\lambda$ is then required in order to compute the difference in binding free energy.
Consequently, whereas in ESMACS all replicas sample using the same system description,
in TIES sub-ensembles are executed at diffent points along $\lambda$.
The change in free energy associated with the transformation is calculated by numerically integrating (via adaptive quadrature) the
values of the $dU/d\lambda$ across the full set of $\lambda$ windows simulated.
The TIES protocol, for example, relies on the numerical
integration of $dU/d\lambda$. 
%Obtaining values for the function means running
%multiple replicas of full simulation workflows with a given $\lambda$,
%therefore we would like to minimize the number of $\lambda$ values that need to 
%be evaluted for our given accuracy threshold. 

Obtaining accurate and precise results from TIES requires that the $\lambda$ windows correctly
capture the changes of $dU/d\lambda$ over the transformation.
This behaviour may vary considerably between systems.
Typically, windows are evenly spaced between 0 and 1 with the spacing between them set before
execution at a distance determined by the simulator to be sufficient for a wide range of systems.
Typically, a TIES ensemble 65 replicas evenly distributed between 13 $\lambda$ windows
In order to obtain a meaningful TIES result it is necessary to not only simulate the drug pair
in the protein but also in an aqueous environemnt, adding a further 65 replicas albeit it using a
smaller system at lower computational cost.

Following simulation (and in the case of ESMACS free energy analysis steps) both protocols require
the execution of short serial steps to provide summary statistics.
Both protocols can are highly customizable, for example the number of simulation replicas in the
ensemble and the lengths of their runs can be varied.
More generally though, there may be cases where it is important to increase the sampling of phase
space possibly through expanding the ensemble or in TIES changing the distributon of $\lambda$
windows.
In addition, the ESMACS protocol can also be extended to account for adaptation energies involved
in altering the conformation of the protein or ligand during binding through the use of separate
component simulations.

%Adaptive quadrature is a numerical integration algorithm that is designed to
%minimize the number of function evaluations while still achieving a given error
%tolerance when estimating the integral. 
%We use this algorithm to minimize the
%number of full simulations that are required to be run, while still keeping the
%high accuracy of the protocol.

\subsection{The Value of Adaptivity}

Both ESMACS and TIES have been sucessfully used to predict binding affinities
quickly and accurately. 
Nonetheless, they are very expensive computationally and
optimizing the execution time while still retaining (or even improving) the accuracy is desirable. 
Drug discovery programmes usually have limited resources, and 
require a resource budget carefully looking to gain maximum insight 
into potential candidate compounds.
This provides one clear motivation for the use of adaptive methods 
which minimize the compute time used whilst producing binding free energy estimates 
meeting pre-defined quality criteria (such as convergence or 
statistical uncertainty below a given theshold).

A second driver for adaptivity is that such programmes will typically involve compounds with 
a wide range of chemical properties which can impact not only 
the time to convergence but the type of sampling required to gain
accurate results.
In general, there is no way to know before running calculations 
exactly which setup of calculation is required for a particular 
system.
For example in TIES, the number (or the exact location) of the $\lambda$ windows 
that will most impact the calculation are not know \textit{a priori} 
and change between systems.
As multiple simulations must be run for each window sampling with a very high frequency is expensive and impractical.
Furthermore, adaptive placement of $\lambda$ windows is likely to better capture the shape of the $dU/d\lambda$ curve, leading to more accurate and precise results for a given cost.
On occasion alchemical methods may be very slow to converge, 
in such circumstances use of another method, such as ESMACS, 
may be the best option.
This means that the most effective way to gain accurate and 
precise free energy results on industrially or clinically 
relavant timescales is to be able to adapt both sampling 
and even the protocol used at run time.
With potentially thousands of simulations, often employing 
multiple analysis methodologies, this is indeed the only way to 
effectively use these techniques at scale.


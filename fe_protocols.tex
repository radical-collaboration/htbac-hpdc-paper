We have designed two free energy calculation protocols with the demands of clinical
decision support and drug design applications in mind: ESMACS (enhanced
sampling of molecular dynamics with approximation of continuum
solvent)~\cite{Wan2017brd4} and TIES (thermodynamic integration with enhanced
sampling)~\cite{Bhati2017}. 
The former protocol is based on variants of the molecular mechanics Poisson-Boltzmann 
surface area (MMPBSA) end-point method, the latter on the `alchemical' thermodynamic 
integration (TI) approach. 
In both cases, ensembles of MD simulations are employed to perform averaging and
to obtain tight control of error bounds in our estimates. 
In addition, the ability to run replica simulations concurrently means that, as long as
sufficient compute resources are available, turn around times can be significantly 
reduced compared to the generation of single long trajectories.
Due to their shared philosophical underpinning both protocols share similar middleware 
requirements.



Each replica within the ESMACS protocol consists of a series of simulation
steps followed by post production analysis. 
Generally, an ESMACS replica will contain between 3 and 12 equilibration 
simulation steps followed by a production MD run, all of which are conducted in 
the NAMD package~\cite{Phillips2005}. 
The first step is system minimization, the following steps involve the gradual release 
of positional constraints upon the structure and the heating to a physiologically realistic 
temperature. 
Upon completion of the MD simulation, free energy computation (via MMPBSA and potentially 
normal mode analysis) is performed using AmberTools~\cite{amber14, Case2005, MillerIII2012}.

The ESMACS protocol is highly customizable. Both the number of simulation 
replicas in the ensemble and the lengths of their runs can be varied to
obtain optimal performance for any given system. Using replicas that only
vary in the initial velocities assigned to the atoms of the system we have
defined a standard protocol which prescribes a 25 replica ensemble, each run
consisting of 2 ns of equilibration and 4 ns of production simulation. Our
protocol has produced bootstrap errors of below 1.5 kcal mol$^{-1}$ (despite
replica values varying by more than 10 kcal mol$^{-1}$) for a varied range of
systems including small molecules bound to kinases and more flexible peptide
ligands binding to MHC proteins \cite{Wan2015, Wright2014, Wan2017brd4}.
In these systems, the error we obtained more than halves between ensembles of
5 and 25 replicas but increases in ensemble size have generally produced only
small improvements. More generally though, there may be cases where it is
important to increase the sampling of phase space either through expanding
the ensemble or by considering multiple initial configurations.

The ESMACS protocol can also be extended to account for adaptation energies
involved in altering the conformation of the protein or ligand during
binding. Almost all MMPBSA studies in the literature use the so-called
1-trajectory method, in which the energies of protein-inhibitor complexes,
receptor proteins and ligands are extracted from the MD trajectories of the
complexes alone. The ESMACS protocol can additionally use separate ligand and
receptor trajectories to account for adaptation energies, providing further
motivation to deploy the protocol via flexible and scalable middleware.
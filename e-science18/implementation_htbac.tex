In \S\ref{ssec:esm_ties} we define the structure of the ESMACS and TIES
protocols. Here we provide skeletons of the TIES protocol implemented in
HTBAC\@. In L.~\ref{lst:ties.py} we show a customization of a production
MD simulation step.

\lstinputlisting[language=Python, label={lst:ties.py}, caption={TIES protocol
implemented with HTBAC. We import the predefined protocol `TIES'. We assign
the physical system to the protocol, we instantiate a simulation, customize
its steps (\texttt{replica}, \texttt{lambda}) and assign it to the TIES's
\texttt{step0}. We instantiate a Runner with a resource request and pass the
protocol to it.}]{ties.py}

In \S\ref{ssec:adaptive_execution} we show HTBAC's adaptive execution
capabilities. In L.~\ref{lst:ties_adaptivity.py} we provide an
intra-protocol adaptive implementation of TIES, based on the use-case
of \S\ref{ssec:adapt_ties}.

\lstinputlisting[language=Python, label={lst:ties_adaptivity.py},
caption={Adaptive TIES protocol implemented with HTBAC. Assuming
L.~\ref{lst:ties.py}, we run the runner retrieving runtime results, we
specify an adaptivity script for the evaluator, create TIES's \texttt{step1}.
The analysis script operates on partial simulation results, generating new
simulation conditions for the next simulation step.}]{ties_adaptivity.py}

% \jhanote{As written, this sub-section has nothing about HTBAC and would be
% better placed/merged in Science Driver.  What would be useful is
% pseudo-code or code listing to show how ESMACS and TIES are encoded in
% HTBAC}\jdnote{added context}

% While ESMACS and TIES protocols compute binding affinity calculations for
% different systems and purposes, their implementation in HTBAC have the same
% underlying pattern, consisting of simulations steps, followed by one or
% more analysis step(s). Designers of free energy protocols can utilize the
% simulation and analysis components of HTBAC to create any customized
% sequence of simulation(s) and/or analysis steps. The main implementation
% differences between protocols is in their design of ensemble members.

% Listing~\ref{lst:esmacs.py} shows the skeleton of the ESMACS protocol
% implemented in HTBAC. Specifically, the listing shows the customization of
% simulation conditions for a minimization step of the protocol.

% In \S\ref{ssec:adaptive_execution} we show HTBAC's adaptive execution
% capabilities. Here we show an example of a 5 step simulation run. For
% brevity, we did not specify the simulation conditions for
% \texttt{TIES.step0--2}, which are typically minimization and equilibration
% steps. We focus on implementing adaptivity within the protocol for the
% production MD step. In a non-adaptive protocol, the production MD step
% executes for the entire simulation duration as specified by the user. The
% adaptive protocol breaks down the production MD step into multiple, shorter
% steps. Between steps, the user assigns analysis scripts that generate
% simulation conditions necessary for further simulations. The shorter
% \texttt{TIES.step3} and \texttt{TIES.step4} specify production MD phase 1
% and phase 2, respectively. Simulations from \texttt{TIES.step3} execute but
% the application does not terminate, as specified by the flag
% \texttt{terminate = false}. The adaptive quadratures function computes
% where to place subsequent $\lambda$ windows and provides the information to
% \texttt{TIES.step4}.

% The additional windows

Ensemble-based protocols provide are showing considerable predictive potential 
% a method for producing robust estimates of binding free energy, 
in computational drug campaigns. As drug screening can cover millions of
compounds and hundreds of millions of core-hours, it is important for
binding affinity calculations to optimize the accuracy and precision of
results obtained from a given set of resources. However, the optimal protocol
for a given system is difficult to determine {\it a priori}, thus requiring
runtime adaptation to workflow executions. We introduce HTBAC to enable 
scalable and adaptive binding affinity energy calculations on HPC. 
% However, variations in the chemical and biophysical properties of different
% systems impact the optimal protocol choice for different proteins and
% classes of drugs targeting them.

Specifically, this paper makes the following contributions: (1) shows
adaptive approaches within a ensemble-based free energy protocol (TIES) to
improve binding affinity accuracy given a fixed amount of computing
resources; (2) characterizes HTBAC, the software systems we developed to address
the aforementioned requirements;
%, the software system we developed to
% enable the scalable execution of adaptive applications; 
and (3) shows the capability to execute adaptive applications at scale, 
validating their scientific results.

We characterize the performance of HTBAC on NCSA Blue Waters. We show
near-ideal weak and strong scaling behavior for ESMACS and TIES, individually 
and together, reaching scales of 21120 cores. 
% Furthermore, we provide preliminary
% strong scaling results using ESMACS and TIES individually, demonstrating
% linear speedup, and consistent overhead. 
Furthermore, we validate binding free energies computed with HTBAC 
with both experimental and previously published computational results.

% We show ideal weak scaling behavior reaching scales of 21120 cores.

We compare computational consumption and free energy accuracy in our adaptive
and non-adaptive TIES results. Using the adaptive quadrature algorithm, we
show improvements in $\Delta \Delta$G on average by 77\% over the 5 physical
systems tested. By reducing the $\lambda$ windows on average by 32\%, we
reduce execution time by the same amount. The adaptive
termination implementation of the TIES protocol saves compute resources and
reduces time to solution on average by 16\%. To the best of our knowledge,
adaptive TIES protocols have not been benchmarked against non-adaptive
implementations before.

% In this paper we demonstrate: (i) How HTBAC allows the concurrent screening
% for drug binding affinities of multiple compounds at unprecedented scales,
% both in the number of candidates and resources utilized. Specifically, we
% investigated weak scaling behavior for screening 8 drug candidates
% concurrently on more than 16,000 cores. This permits a rapid
% time-to-solution that is essentially invariant with respect to the
% calculation protocol, size of target system and number of ensemble
% simulations. (ii) The validation of binding free energies computed using
% HTBAC with both experimental and previously published computational
% results; (iii) HTBAC enabled the adaptive execution of the TIES protocol
% providing greater convergence (i.e., lower errors) for a given amount of
% computational resources. To the best of our knowledge, adaptive TIES
% protocols have not been benchmarked against non-adaptive implementations.

% nor have they been implemented at such scales before.

% HTBAC can also support a wide range of adaptivite scenarios.

% As such, HTBAC advances the state of the scale and sophistication of
% binding affinity calculation. In addition to reporting increasingly
% sophisticated adaptive scenarios, in future, we will extend HTBAC to
% support the ``design of experiments", facilitating optimization at the
% level of the overall computational campaign and time-to-insight for a large
% database of drug candidates, as opposed to for single simple calculations.

% HTBAC uses readily available building blocks to attain both workflow
% flexibility and performance; our scaling experiments are performed on the
% Blue Waters machine at NCSA.
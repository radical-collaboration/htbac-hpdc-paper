\jhanote{A section with a single subsection is bad style. I think this can be
compressed and the need for a subsection removed. See related\_work2.tex}

The strength of drug binding is determined by a thermodynamic property known
as the binding free energy (or binding affinity). One promising technology for
estimating binding free energies and the influence of protein and ligand
composition upon them is molecular dynamics (MD) ~\cite{Karplus2005}. A
diversity of methodologies have been developed to calculate binding affinities
MD sampling~\cite{Mobley2012} and blind tests show that many have considerable
predictive potential~\cite{Mey2017, Yin2017}. We have designed two protocols
with the demands of clinical decision support and drug design applications in
mind: ESMACS (enhanced sampling of molecular dynamics with approximation of
continuum solvent)~\cite{Wan2017brd4} and TIES (thermodynamic integration with
enhanced sampling)~\cite{Bhati2017}. The former protocol is based on variants
of the molecular mechanics Poisson-Boltzmann surface area (MMPBSA), which is
an `approximate' end-point method~\cite{Massova1999}. The latter on the
rigorous `alchemical' thermodynamic integration (TI)
approach~\cite{Straatsma1988}.

Using these protocols, we have demonstrated the lack of reproducibility of
individual simulations for a variety of protein systems, with calculations for
the same protein-ligand combination using almost identical initial conditions
producing widely varying results (binding affinities varying by up to 12 kcal
mol $^{-1}$ for small molecules, whilst flexible ligands can vary even
more)~\cite{Wan2015, Sadiq2010, Wright2014}. Indeed, our work has revealed how
completely unreliable single simulation based approaches are. However, we have
shown that averaging across multiple runs can reliably produce results in
agreement with previously published experimental findings~\cite{Sadiq2010,
Wan2011, Wright2014, Bhati2017, Wan2017brd4, Wan2017trk}, and correctly
predicted the results of experimental studies performed by colleagues in
collaboration~\cite{Bunney2015}. We term this approach ensemble molecular
dynamics, ``ensemble'' here referring to the set of individual (replica)
simulations conducted for the same physical system. In this Section we discuss
the advantages to this approach.

% ---------------------------------------------------------------------------
\subsection{Ensemble Molecular Dynamics Simulations}

Atomistically detailed models of the drug and target protein can be used as the
starting point for MD simulations to study the influence of changes in 
either drug or protein composition on drug binding. 
The chemistry of the system is encoded in what is known as a potential
~\cite{Karplus2002}. 
In the parameterization of the models, each atom is assigned a mass and a charge,
with the chemical bonds between them modeled as springs with varying
stiffness.
Newtonian mechanics is then used to follow the dynamics of the protein and drug and, 
using the principles of statistical mechanics, estimates of thermodynamic properties 
obtained from simulations of single particles.

The potentials used in the simulations are completely under the control of
the user. 
This allows the user to manipulate the system in ways which would
not be possible in experiments. 
A particularly powerful example of this are
the so called ``alchemical'' simulations in which the potential used in the
simulation changes, from representing a particular starting system into one 
describing a related target system during execution. 
This allows for the calculation of free energy differences between the two 
systems, such as those induced by a protein mutation.

MD simulations can reveal how interactions change as a result of chemical alterations,
and account for the molecular basis of drug efficacy. 
This understanding can form the basis for structure-based drug design as well as helping 
to target existing therapies based on protein sequence. 
However, correctly capturing the system behavior poses at least two major 
challenges: The approximations made in the potential must be accurate enough representations 
of the real system chemistry; and sufficient sampling of phase space is also required.
Our previous work demonstrated \cite{Sadiq2010, Wan2011} that running multiple 
MD simulations based on the same system and varying only in initial velocities
offers a highly efficient method of obtaining accurate and reproducible
estimates of the binding affinity.
While the accuracy of force fields could be a source of error, we know from our work to 
date that the very large fluctuations in trajectory-based calculations account for the lion’s 
share of the variance (hence also uncertainty) of the results.

In order for MD simulations to attain widespread use in industrial (and potentially 
clincal) settings, it is necessary that results can be obtained in a timely fashion.
This typically means that results are required in a mattre of days (or at most a week) 
in order that the results can feed into decision making processes.
The necessity for rapid turn around times places additional demands on
simulation protocols which need to be optimized to gain results with a short
turn around time. 
Further to the scientific and practical considerations
outlined above, it is vital that reliable uncertainty estimates are
provided alongside all quantitative results for simulations to provide
actionable predictions.

We have developed a number of free energy calculation protocols based on the
use of ensemble molecular dynamics simulations with the aim of meeting these
requirements~\cite{Sadiq2008, Sadiq2010, Wan2017brd4, Wan2017trk}.
Basing these computations on the direct calculation of ensemble averages facilitates
the determination of statistically meaningful results along with complete
control of errors. 
As we scale up to larger datasets the use of the ensemble approaches will increasingly 
necessitate the use of middleware to  provide reliable coordination and distribution 
mechanisms with low performance overheads.

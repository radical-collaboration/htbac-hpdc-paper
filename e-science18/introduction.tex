\textbf{Start with importance of binding free energy for drug design}

Binding free energy calculations rely on the physics-based molecular simulations 
and statistical mechanisms in order to predict structure-based drug design. 
Based on recent improvements in the accuracy of binding free energy, free energy 
calculations are now becoming reliable methods in the drug discovery pipeline. 
These improvements can be attributed to many advances in methodologies, as well 
as enhancements in hardware. Specifically, ensemble-based 
binding free energy calculations, which favor multiple shorter simulation 
trajectories over few longer simulations are showing improvements in biological 
time scales and can now leverage binding affinity calculations to provide 
meaningful clinical insight into computational drug campaigns and potentially 
reduce the cost of expensive experimental screening programs. 
These approaches could also be used to determine the influence of individual 
patients genetic sequence on drug efficacy in the context of clinical decision 
support. In order for these approaches to gain traction, requires using 
simulation protocols which have well-defined uncertainty and consistently 
produce statistically meaningful results. 

\textbf{2 prominent free energy protocols} 

Computational drug campaigns rely on rapid screening of thousands of compounds. 
At the start of a drug design campaign an initial screening of candidate 
compounds is conducted to filter out the worst binders before using more 
sensitive methods to refine the structure. Two prominent ensemble-based 
free energy protocols, ESMACS and TIES~\cite{Bhati2017} have demonstrated the 
ability to filter and refine the drug design process. The ESMACS (enhanced 
sampling of molecular dynamics with approximation of continuum solvent) protocol 
provides an ``approximate'' endpoint method used to screen out poor binders. The 
TIES protocol (thermodynamic integration with enhanced sampling) uses a more 
rigorous ``alchemical" thermodynamic integration approach. These protocols have 
demonstrated statistically meaningful results at timescales relevant for 
industrial computational drug campaign. 

\textbf{Scalable computing approaches that try to utilize CPU hours and cores}

Scalable simulation computing approaches such as ensemble-based free energy 
protocols in drug campaigns are utilizing the growing number of compute 
resources by scaling the number of highly parallel simulations. Most of these 
computational approaches focus on the utilization of core hours using 
fire-and-forget executions, but not many try to use core-hours effectively. 

\textbf{...But not many approaches try to use core-hours effectively}

To address the rapidly growing screening process of production scale drug 
campaigns which require screening of millions of compounds, tools that only 
leverage the scalable computation of varied binding free energy calculations on 
HPC resources are not enough to deliver on timescales that are clinically 
warranted. 

\textbf{In order to do so requires advances in methods and resource efficiency}

Moreover, free energy binding protocols need better sophistication to 
maximize resources for simulations that yield the best accuracy and precision 
given set of resources or fixed computing time. However, computational drug 
campaigns screen for compounds that are unique in chemical properties. The main 
challenge lies in the variability of statistical uncertainty of the free energy 
binding between compounds. Ensemble-based free energy protocols leverage 
generality of chemical properties to maintain convergence of binding free energy 
across compounds with variable degrees of freedom, but at the cost of 
inefficiency in resources consumption. Moreover, generality also means that 
once accuracy thresholds are reached, the protocol has converged, which prevents
protocols from achieving better accuracy for certain compounds. \jdnote{needs a 
stronger motivation} One of the 
difficulties involved in defining adequate protocols is that differences in 
drug candidate chemistry (or protein sequence) can alter simulation behavior and 
the convergence of binding strength estimates. For example, not all changes 
induced in protein shape or behavior are local to the drug binding site and, in 
some cases, simulation protocols will need to adjust to account for complex 
interactions between drugs and their targets within individual studies. Advances 
in methodologies can provide the ability to capture complexities in simulations 
for a given compound, however these methodologies rely on learning the positions 
of simulations during execution.\jdnote{alternatively we could include this 
instead of the last sentence of this paragraph: A more strategic concern is that 
during a study the requirements may change.} 
 

\textbf{Specific adaptive methods within binding free energy can enhance 
clinical insight}

Here we show how adaptive approaches within ensemble-based free energy 
protocols are designed to capture unique chemical properties and thereby 
customize the simulations for a candidate in a way that makes the most 
effective use of computational resources in order to deliver better fidelity of 
statistical uncertainties than general approaches. For drug campaigns, we can 
leverage adaptive approaches to support drug campaigns for clinical insight. 

\textbf{Suggest solution to address execution of scalable adaptive methods}

\jdnote{following paragraphs are taken from EnTK, will fix}
Most tools do not support the ability to encode and execute more complex 
algorithm logic, let alone provision runtime capabilities that can change 
resource configurations based on intermediately generated data.  
We need a tool that can leverage scalable execution of adaptive algorithms.

Implementation of adaptive algorithms on high performance clusters are 
predicated on adaptive runtime systems that provide the ability to make 
runtime decisions based on intermediate results and can manage resources 
efficiently. To achieve scalability and efficiency, such adaptivity cannot be 
performed via user intervention and hence automation of the control logic and 
execution is important. HTBAC enables the scalable execution of adaptive 
algorithms. 


% Leveraging ensemble-based simulations, we can now explore new adaptive sampling 
% schemes can improve sampling quality of ligand binding affinity of individual drug candidates

% \textbf{Different adaptivity schemas exist, highlight intra-protocol adaptivity but other forms of adaptivity exist}

This paper makes the following contributions:
\begin{itemize}
  \item Identifies the challenge in advancing adaptive algorithms and 
  methodologies
  \item Shows the importance of using adaptive approaches within ensemble-based
  free energy protocols to improve binding affinity accuracy given a fixed 
  amount of computing
  \item Provides the adaptive software solution (HTBAC) that enables the 
  scalable execution of adaptive algorithms
  \item Demonstrates the capability to execute adaptive applications at scale 
  and validates the scientific results from these applications.
\end{itemize}

\textbf{results of scaling experiments and intra-protocol adaptivity experiments}

This paper is organized as follows: 
Section~\ref{sec:science-drivers} describes ESMACS and TIES as ensemble-based 
free energy protocols as well as an adaptive methodology that is implemented 
within the protocol that motivates the need for higher precision given limited 
resources. 
Section~\ref{sec:related-work} describes the motivation for ensemble-based 
approaches and existing solutions and their limitations in their ability to 
support adaptive methods.  
Section~\ref{sec:htbac} describes the design and implementation of 
HTBAC--the software tool that addresses the requirements of scalable, 
adaptive methods. 
In Section~\ref{sec:experiments} we demonstrate experiments of scalability, 
validation of results, and adaptive simulation methods. We show that given a 
fixed amount of computing resources, we can achieve better accuracy and hence 
better time to solution using adaptive simulation methods. 






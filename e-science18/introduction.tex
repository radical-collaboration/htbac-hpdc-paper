% \textbf{Start with importance of binding free energy for drug design}

Drug discovery and design is immensely expensive with one study putting the
current cost of each new therapeutic molecule that reaches the clinic at
US\$1.8 billion~\cite{Paul2010}. Furthermore, the basic premise of
one-size-fits-all treatments has been eroded by the need for stratified and
even personalized treatment regimes. Computational approaches, specifically
binding free energy calculations which rely on physics-based molecular
dynamics simulations (MD) and statistical mechanics, support a better
understanding of structure-based drug design at a vastly reduced cost
compared to wet lab alternatives.

% Recent improvements in binding free energy calculations mean that they are
% now becoming reliable methods in the drug discovery pipeline.

% One promising technology for estimating binding free energies and the
% influence of protein and ligand composition upon them is molecular dynamics
% (MD)~\cite{Karplus2005}.

A diversity of methodologies have been developed to calculate binding
affinities MD sampling~\cite{Mobley2012} and blind tests show that many have
considerable predictive potential~\cite{Mey2017,Yin2017}. The development of
commercial approaches that claim accuracy of below 1 kcal
mol$^{-1}$~\cite{Wang2015} has led to increased interest from the
pharmaceutical industry~\cite{Ganesan2017} in designing computational drug
campaigns.

% The same technologies have also shown promise in analysing the influence of
% protein mutations on drug binding~\cite{Mondal2016, Bunney2015}. These
% developments have also spurred renewed interest in the factors affecting
% the performance of different free energy methods~\cite{Aldeghi2017,
% Cappel2016, Ruiter2016}.

These improvements can be attributed to many advances in methodologies, as
well as enhancements in hardware. Specifically, ensemble-based binding free
energy calculations, which favor many shorter simulation trajectories over
few longer simulations have been shown to increase sampling efficiency whilst
also reducing time to insight. This means that that such approaches are
becoming a viable alternative to expensive experimental screening programs.
The same techniques are also applicable to gaining clinical insight into the
influence of individual patients genetic sequence on drug efficacy in the
context of clinical decision support. For binding affinity calculations to
gain traction, they have to use simulation protocols which have well-defined
uncertainty and consistently produce statistically meaningful results.

% \textbf{2 prominent free energy protocols}

Computational drug campaigns rely on rapid screening of millions of
compounds, which start with an initial screening of candidate compounds to
filter out the worst binders before using more sensitive methods to refine
the structure of drug design\mtnote{structure of?}\jdnote{addressed}. Two 
prominent ensemble-based 
free energy protocols, ESMACS and TIES~\cite{Bhati2017}, have demonstrated the 
ability to consistently filter and refine the drug design process. The ESMACS 
(enhanced sampling of molecular dynamics with approximation of continuum solvent)
protocol provides an ``approximate'' endpoint method used to screen out poor
binders. The TIES protocol (thermodynamic integration with enhanced sampling)
uses a more rigorous ``alchemical'' thermodynamic integration approach. These
protocols have demonstrated statistically meaningful results at timescales
relevant for industrial computational drug campaign\jdnote{maybe this
requires a citation from UCL?}\mtnote{agreed}.

% \textbf{Scalable computing approaches that try to utilize CPU hours and
% cores}

Scalable simulation computing approaches such as ensemble-based free energy
protocols in drug campaigns are utilizing a growing number of compute
resources by scaling the number of parallel simulations they execute. Most of
these computational approaches focus on the utilization of core hours, using
fire-and-forget executions, but not many try to use these core-hours
effectively. As the scope of drug screening campaigns can be of the order of
millions of compounds, effectiveness is vital for the uptake of binding
affinity calculations in industrial and clinical settings. 

% Such an uptake cannot be supported by tools that only leverage scalable
% computation of varied binding free energy calculations on HPC resources.

% \textbf{...But not many approaches try to use core-hours effectively}
 
% To address the rapidly growing screening process of production scale drug
% campaigns which require screening of millions of compounds, tools that only
% leverage the scalable computation of varied binding free energy
% calculations on HPC resources are not enough to deliver on timescales that
% are clinically warranted.

% \textbf{In order to do so requires advances in methods and resource
% efficiency}

In practical applications, binding free energy protocols must also optimize
the accuracy and precision of results obtained from a given set of resources.
% (for example fixed computing time). 
This is challenging as, by definition, drug discovery involves screening
compounds that are highly varied and potentially unique in their chemical
properties. The variability in the drug candidate chemistry 
% (or protein sequence) 
results in divergent sampling behavior that increases the statistical
uncertainty of binding free energy predictions, making convergence
unpredictable.
% and consequently the convergence and statistical uncertainty of binding
% free energy predictions for a given compounds are unpredictable.
Further, a particular difficulty comes from the fact that not all changes
induced in protein shape or behavior are local to the drug binding site and,
in some cases, simulation protocols will need to adjust to account for
complex interactions between drugs and their targets within individual
studies.

Traditionally, simulation protocols have been deployed with the worst case
scenario in mind and with sufficient sampling conducted to account for the
slowest convergence previously encountered. This approach has at least two
shortcomings: it wastes valuable resource allocation and fails to account
for the different value of the simulations results. For example, in a drug
campaign it is more important to understand how strong the binding of the
best compound candidates is than precisely knowing how weak the interactions
of the worst compound is.

% time. This approach also fails to account for the different value provided
% by each of results, e.g., it is less important to know precisely how weak
% the interactions of the worst compound is compared to how strong the
% binding is of the top candidates.

Key to successful campaigns is identifying when small chemical changes
result in large changes in binding strength. This can mean that the
parameters which are important to campaign evolve as the study progresses.
Here we show how adaptive approaches within ensemble-based free energy
protocols can be designed to capture unique chemical properties and thereby
customize the simulations for a candidate in a way that makes the most
effective use of computational resources to deliver better fidelity of
statistical uncertainties than general approaches.

% Advanced methodologies can capture complexities in simulations for a given
% compound, however they rely on learning the positions of simulations during
% execution.
% \jdnote{alternatively we could include this instead of the last sentence of
% this paragraph: A more strategic concern is that during a study the
% requirements may change.} For drug campaigns, we can leverage adaptive
% approaches to support drug campaigns for clinical insight.

% These approaches can be tailored to support drug design campaigns or
% clinical decision support applications where strategic concerns may be
% different (and even vary over the course of a study).

% Ensemble-based free energy protocols leverage generality of chemical
% properties to maintain convergence of binding free energy across compounds
% with variable degrees of freedom, but at the cost of inefficiency in
% resources consumption. Moreover, generality also means that once accuracy
% thresholds are reached, the protocol has converged, which prevents
% protocols from achieving better accuracy for certain compounds.
% \jdnote{needs a stronger motivation}
 
% \textbf{Specific adaptive methods within binding free energy can enhance
% clinical insight}

% Existing software tools do not support the ability to encode and execute
% more complex algorithm logic that can leverage intermediately generated
% data and provision runtime capabilities that have dynamic resource
% configurations.

Software tools that leverage scalable execution of adaptive algorithms are
currently in demand for supporting efficient drug campaigns. Implementation
of adaptive algorithms on high performance computers (HPC) require adaptive
runtime systems that provide the ability to make runtime decisions based on
intermediate results and can manage resources efficiently. To achieve
scalability and efficiency, such adaptivity cannot be performed via user
intervention and hence automation of the control logic and execution is
important. We have developed the High-Throughput Binding Affinity Calculator
(HTBAC) to address these needs to enable the scalable execution of adaptive
algorithms.

% Leveraging ensemble-based simulations, we can now explore new adaptive
% sampling schemes can improve sampling quality of ligand binding affinity of
% individual drug candidates

% \textbf{Different adaptivity schemas exist, highlight intra-protocol
% adaptivity but other forms of adaptivity exist}

This paper makes four main contributions: (1) identifies the challenge in
advancing adaptive algorithms and methodologies; (2) shows the importance of
using adaptive approaches within ensemble-based free energy protocols to
improve binding affinity accuracy given a fixed amount of computing
resources; (3) presents and characterizes HTBAC, a software tool that enables
the scalable execution of adaptive applications; and (4) shows the capability
to execute adaptive applications at scale, validating their scientific
results.

This paper is organized as follows: Section~\ref{sec:science-drivers}
introduces ESMACS and TIES, two ensemble-based free energy protocols, arguing
how implementing adaptive methodology within TIES could achieve higher
precision with limited resources. Section~\ref{sec:related-work} describes
the motivation for ensemble-based approaches and existing solutions alongside
the limitations in their ability to support adaptive methods.
Section~\ref{sec:htbac} describes the design and implementation of HTBAC and
in and Section~\ref{sec:implementation_htbac} we show how we used HTBAC to
implement ESMACS and an adaptive and nonadaptive version of TIES. In
Section~\ref{sec:experiments} we present experiments we performed with HTBAC
to characterize its scalability and overheads, and showing that given a fixed
amount of computing resources, we can achieve better accuracy and hence
better time to solution using adaptive simulation approaches.
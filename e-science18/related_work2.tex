Free-energy calculations using MD simulations occur in
a wide range of research including protein folding and assessing small molecule 
binding. Free-energy calculation require three main components: 
(1) suitable model Hamiltonian; (2) sampling protocol;
and (3) estimator for free energy. Several approaches to computing binding
free energies exist, amongst which relative binding free energy (or binding
affinity) calculations are generating accurate predictions, delivering
considerable promise for computational drug campaigns~\cite{Karplus2005}.

Ensemble-based simulations have been shown to reduce the sampling time
required to deliver the precision necessary to meet the requirements of drug
design campaigns. Several ensemble-based methods are widely used to compute 
binding free energies, studying different problem spaces. For example, a popular 
approach is to use Markov state models to learn a simplified representation of 
the explored phase space and to choose which regions should be further 
sampled~\cite{Bowman2010}. Replica exchange with solute tempering use the 
Metropolis-Hastings criteria to make periodic decisions about what regions of 
the phase space to sample~\cite{Earl2005,Hritz2008,Kim2012}. In
expanded ensemble simulations, thermodynamic states are explored via a biased
random walk in state space~\cite{Lyubartsev1992}. Approaches that learn by
exchanging information have been found to improve sampling results and
decorrolate as fast or faster than standard simulations.

In relative binding free energy calculations sampling is performed
at discrete regions along the transformation between the two compounds.
The choice of where exactly this sampling occurs is a key
determinant of the uncertainty in and accuracy of the
calculations~\cite{Ruiter2013, Ruiter2016}. Increasing simulations in regions of 
most rapid change reduces errors on the predicated binding affinity.

Executing scalable and adaptive simulation methods directly on High
Performance Computers (HPCs) and other infrastructures using ensemble-based
methods presents several challenges. On HPCs, users can only exploit the
adaptive features provided by MD kernels and often run applications with
execution models such as MPI. For example, a parallel programming systems
such as \texttt{Charm++} relies on MPI to spawn multiple executables of MD
kernels, thereby incurring communication overheads and failures at larger
scales.

By designing ensemble-based methods to use MPI, it is difficult to implement
existing applications in other programming models. This is especially the
case when running ensemble-based methods on clouds. The ensemble-based
methods designed using MPI specifically on HPCs cannot be transported without
rewriting the algorithm to fit into a specific programming model such as
MapReduce.

As we reach higher scales of concurrency required by ensemble-based methods,
the use of middleware becomes a necessity in order to provide reliable
coordination and distribution mechanisms and low performance overheads. Some
existing software tools enable execution of ensemble-based methods on HPCs,
yet only offer end-to-end solutions and lack in providing portability and
flexibility, bounding the user to a particular infrastructure. In
domain-specific projects or languages, the user is limited to
adaptive features that target specific adaptive sampling methods, and require
explicit management of resources.

Domain scientists who wish to create adaptive methods require a solution that
provides the flexibility to create ensemble-based free energy
calculations with the ability to incorporate user-defined adaptivity. These
considerations provide the motivation behind the design and development of
HTBAC which serves as a software tool to enable ensemble-based free energy
methods to run adaptively and at scale on HPC resources.

% \begin{itemize}
%   \item Binding affinity calculations in general
%   \item Binding affinity using ensemble approach
%   \item How are people doing scalable binding affinity calculations
%   currently
%   \item Overview of adaptive algorithms used for free energy calculations
%   i.e. MSM, REST, EE
%   \item How are adaptive algorithms managed
%   \item What are the existing tools: HPCs: MPI mainly, clouds and
%   ensemble-based middleware
%   \item Appreciate why ensemble-based methods are not done in clouds
% \end{itemize}

% In this approach the Hamiltonian used in the calculation is gradually
% transformed from a description of an initial compound into that of a second
% molecule and the difference in binding strength between the two is
% computed.

% The improvements in hardware, namely GPGPUs, are leveraging more rapid
% simulations, however to deliver clinical insight for drug screening
% campaigns requires more rapid timescales and better binding precision...
% \jdnote{EnTK paper mentions D.E. Shaw paper on specialized hardware and
% running single long simulation}

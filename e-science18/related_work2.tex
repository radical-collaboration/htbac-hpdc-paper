Free-energy calculations using MD simulations occur in a wide range of
research including protein folding and assessing small molecule binding.
Free-energy calculations require three main components: (1) suitable
Hamiltonian model; (2) sampling protocol; and (3) estimator of free energy.
Several approaches to computing binding free energies exist, amongst which
relative binding free energy (or binding affinity) calculations are
generating accurate predictions, delivering considerable promise for
computational drug campaigns~\cite{Karplus2005}.

Ensemble-based simulations have been shown to reduce the sampling time
required to deliver the precision necessary to meet the requirements of drug
design campaigns. Several ensemble-based methods are widely used to compute
binding free energies, studying different problem spaces. For example, a
popular approach is to use Markov state models to learn a simplified
representation of the explored phase space and to choose which regions should
be further sampled~\cite{Bowman2010}. Replica exchange with solute tempering
use the Metropolis-Hastings criteria to make periodic decisions about what
regions of the phase space to sample~\cite{Earl2005,Hritz2008,Kim2012}. In
expanded ensemble simulations, thermodynamic states are explored via a biased
random walk in state space~\cite{Lyubartsev1992}. Approaches that learn by
exchanging information have been found to improve sampling results and
decorrelate as fast or faster than standard simulations.

In binding affinity calculations sampling is performed at discrete regions
along the transformation between the two compounds. The choice of where
exactly this sampling occurs is a key determinant of the uncertainty in and
accuracy of the calculations~\cite{Ruiter2013,Ruiter2016}. Increasing
simulations in regions of most rapid change reduces errors on the predicated
binding affinity.

Using ensemble-based methods to compute binding affinities of a large number
of drug candidate involves a hierarchy of computational processes: at the
lowest level is the specific molecular dynamics (MD) simulation using an MD
engine, such as Gromacs or AMBER. An ensemble-based \textbf{algorithm} (or
equivalently \textbf{protocol}) is comprised of multiple such MD simulations
that are collectively used to compute the binding free energy of a single drug
candidate. There are multiple protocols that can be used, each comes with its
specific trade-offs. For example, TIES and ESMACS are two protocols to compute
binding affinities that differ in their accuracy but also their computational
cost. The computational instance implementing a protocol with specific
parameter values, number of simulations and other computational aspects of
that protocol, constitutes a \textbf{workflow}. A workflow may be fully
specified a priori, or it may adapt one or more of its properties, say parameters,
as a consequence of intermediate results. Typically, there is a 1-to-many
relationship between protocol and workflows and different workflows can be
used to compute a given binding affinity calculation for a given drug
candidate.

When multiple drug candidates need to be evaluated with certain constraints
and a defined objective, the entire computational activity (i.e., computing
binding affinities for multiple drug candidates) is referred to as a
\textbf{computational campaign}. The objective of the computational campaign
of relevance to this paper is to maximize the number of drug candidates for
which the binding affinity of each individual candidate is determined to
within a (given) acceptable level of error. The campaign is constrained by
the computational resources available, measured in thousand of core-hours. To
meet this objective, each workflow computing the binding affinity of a drug
candidate is adaptively executed.

Executing scalable and adaptive simulation methods on production-grade HPC
resources using ensemble-based methods presents several
challenges~\cite{cosb18,adaptivebiomolecular}. HTBAC addresses these
challenges in the context of a computational campaign to compute
ensemble-based free energy methods.
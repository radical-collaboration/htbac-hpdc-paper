\begin{itemize}
  \item Binding affinity calculations in general
  \item Binding affinity using ensemble approach
  \item How are people doing scalable binding affinity calculations currently
  \item Overview of adaptive algorithms used for free energy calculations i.e. 
  MSM, REST, EE 
  \item How are adaptive algorithms managed
  \item What are the existing tools: HPCs: MPI mainly, clouds and 
  ensemble-based middleware 
  \item Appreciate why ensemble-based methods are not done in clouds
\end{itemize}

Free-energy calculations using molecular dynamics simulations are now used in 
a wide range of research including protein folding and interactions as well as
the assement of small molecule binding. 
Free-energy calculation require 3 main components, a suitable model Hamiltonian,
sampling protocol and estimator for free energy. 
Several approaches to computing binding free-energies exist, amongst which relative 
binding free energy (or binding affinity) calculations are generating accurate 
predictions delivering considerable promising for hit-to-lead optimization. 
In this approach the Hamiltonion used in the calculation is gradually transformed 
from a description of an intial compound into that of a second molecule and the 
difference in binding strength between the two computed.

%The improvements in hardware, namely GPGPUs, are leveraging more rapid 
%simulations, however to deliver clinical insight for drug screening campaigns 
%requires more rapid timescales and better binding precision...
%\jdnote{EnTK paper mentions D.E. Shaw paper on specialized hardware and 
%running single long simulation}

Ensemble-based simulations, have been shown to reduce the sampling time required to 
to deliver the precision necessary to meet the requirements of drug design campaigns.
In biomolecular simulation, several classes of ensemble-based algorithms are widely used 
in the computation of binding free energies. 
Each exist to tackle different problem spaces. 
For example, an increasingly popular approach is to
used Markov state models (MSM) to learn a simplified representation of the
phase space explored by simulations and use this to inform which regions are
further sampled in future simulations \cite{Bowman2010}.
Other adaptive methods such as replica exchange and well-known variations of
replica exchange i.e. replica exchange with solute tempering (REST) use the
Metropolis-Hastings criteria to make periodic decisions that impact which
regions of the phase space are being sampled \cite{Earl2005, Hritz2008, Kim2012}.
In expanded ensemble simulations, thermodynamic states are explored via a biased 
random walk in state space \cite{Lyubartsev1992}.
Approaches such as these that learn by exchanging information have been found to
improve sampling results, and always decorrelating as fast or faster than standard 
simulations.

Generally in relative binding free energy calculations sampling is performed at 
discrete regions along the transformation between the two compounds.
Consequently, the choice of where exactly this sampling occurs is a key determinant 
of the uncertainty in and accuray of the calculations \cite{Ruiter2013}.
Increasing simulation in regions of most rapid change reduces errors on the 
predicated binding affinity. 

Executing simulations directly on HPCs using ensemble-based methods presents 
several challenges. 
Most users rely on using MD kernels directly with 
execution models such as MPI to run simulations. 

Why is running adaptive ensemble-based simulations on clouds a big faux-pas?

Some existing workflow tools enable scalable execution of ensemble-based 
approaches on HPCs, yet only offer end-to-end solutions...\jdnote{BAC is one 
that is domain specific, we have others}


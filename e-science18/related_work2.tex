% \begin{itemize}
%   \item Binding affinity calculations in general
%   \item Binding affinity using ensemble approach
%   \item How are people doing scalable binding affinity calculations currently
%   \item Overview of adaptive algorithms used for free energy calculations i.e. 
%   MSM, REST, EE 
%   \item How are adaptive algorithms managed
%   \item What are the existing tools: HPCs: MPI mainly, clouds and 
%   ensemble-based middleware 
%   \item Appreciate why ensemble-based methods are not done in clouds
% \end{itemize}

Free-energy calculations using molecular dynamics simulations are now used in 
a wide range of research including protein folding and interactions as well as
the assessment of small molecule binding. Free-energy calculation require 
3 main components, a suitable model Hamiltonian, sampling protocol and estimator 
for free energy. Several approaches to computing binding free-energies exist, 
amongst which relative binding free energy (or binding affinity) calculations 
are generating accurate predictions delivering considerable promise for 
computational drug campaigns~\cite{Karplus2005}. 

% In this approach the Hamiltonian used in the 
% calculation is gradually transformed from a description of an initial compound 
% into that of a second molecule and the difference in binding strength between 
% the two is computed.

%The improvements in hardware, namely GPGPUs, are leveraging more rapid 
%simulations, however to deliver clinical insight for drug screening campaigns 
%requires more rapid timescales and better binding precision...
%\jdnote{EnTK paper mentions D.E. Shaw paper on specialized hardware and 
%running single long simulation}

Ensemble-based simulations, have been shown to reduce the sampling time 
required to deliver the precision necessary to meet the requirements of 
drug design campaigns. In biomolecular simulation, several classes of 
ensemble-based methods are widely used in the computation of binding free 
energies. Each exist to tackle different problem spaces. For example, an 
increasingly popular approach is to used Markov state models (MSM) to learn a 
simplified representation of the phase space explored by simulations and use 
this to inform which regions are further sampled in future simulations 
\cite{Bowman2010}. Other adaptive methods such as replica exchange and 
well-known variations of replica exchange i.e. replica exchange with solute 
tempering (REST) use the Metropolis-Hastings criteria to make periodic decisions 
that impact which regions of the phase space are being sampled 
\cite{Earl2005, Hritz2008, Kim2012}. In expanded ensemble simulations, 
thermodynamic states are explored via a biased random walk in state space 
\cite{Lyubartsev1992}. Approaches such as these that learn by exchanging 
information have been found to improve sampling results, and always 
decorrelating as fast or faster than standard simulations.

Generally in relative binding free energy calculations sampling is performed at 
discrete regions along the transformation between the two compounds.
Consequently, the choice of where exactly this sampling occurs is a key 
determinant of the uncertainty in and accuracy of the calculations 
\cite{Ruiter2013}. Increasing simulation in regions of most rapid change reduces 
errors on the predicated binding affinity. 

Capturing and executing scalable and adaptive simulations methodologies 
directly on High Performance Computers (HPCs) and other infrastructures using 
ensemble-based methods presents several challenges. On HPCs, users can only  
exploit the adaptive features provided by MD kernels and often run applications with
execution models such as MPI. For example, a parallel programming systems such 
as Charm++ relies on MPI to spawn multiple executables of MD kernels, thereby 
incurring communication overheads and failures at larger scales. By designing 
ensemble-based methods to use MPI, it is difficult to implement existing 
applications in other programming models. 

This is especially the case when running ensemble-based methods on clouds. 
The ensemble-based methods designed using MPI specifically on HPCs cannot be 
transported without rewriting the algorithm to fit into a specific programming 
model such as MapReduce. 

As we reach higher scales of concurrency required by ensemble-based methods,
the use of middleware becomes a necessity in order to provide reliable
coordination and distribution mechanisms and low performance overheads. Some 
existing software tools enable execution of ensemble-based methods on HPCs, 
yet only offer end-to-end solutions and lack in providing portability and 
flexibility bounding the user to a particular infrastructure. In domain-
specific projects or languages, the user is bound to a limited 
adaptive features that target specific adaptive sampling methods, and require 
explicit management of resources.

Domain scientists who wish to create adaptive methods require a solution that 
provides the flexibility to create and customize ensemble-based free energy 
calculations with the ability to incorporate user-defined adaptivity. These 
considerations provide the motivation behind the design and development of
HTBAC which serves as a software tool to enable ensemble-based free energy 
methods to run adaptively and at scale on HPC resources. 










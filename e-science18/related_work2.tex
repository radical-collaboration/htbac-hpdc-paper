Free-energy calculations using MD simulations occur in
a wide range of research including protein folding and assessing small molecule 
binding. Free-energy calculation require three main components: 
(1) suitable model Hamiltonian; (2) sampling protocol;
and (3) estimator for free energy. Several approaches to computing binding
free energies exist, amongst which relative binding free energy (or binding
affinity) calculations are generating accurate predictions, delivering
considerable promise for computational drug campaigns~\cite{Karplus2005}.

Ensemble-based simulations have been shown to reduce the sampling time
required to deliver the precision necessary to meet the requirements of drug
design campaigns. Several ensemble-based methods are widely used to compute 
binding free energies, studying different problem spaces. For example, a popular 
approach is to use Markov state models to learn a simplified representation of 
the explored phase space and to choose which regions should be further 
sampled~\cite{Bowman2010}. Replica exchange with solute tempering use the 
Metropolis-Hastings criteria to make periodic decisions about what regions of 
the phase space to sample~\cite{Earl2005,Hritz2008,Kim2012}. In
expanded ensemble simulations, thermodynamic states are explored via a biased
random walk in state space~\cite{Lyubartsev1992}. Approaches that learn by
exchanging information have been found to improve sampling results and
decorrolate as fast or faster than standard simulations.

In binding affinity calculations sampling is performed at discrete regions
along the transformation between the two compounds. The choice of where
exactly this sampling occurs is a key determinant of the uncertainty in and
accuracy of the calculations~\cite{Ruiter2013, Ruiter2016}. Increasing
simulations in regions of most rapid change reduces errors on the predicated
binding affinity.

Computing the binding free energy of a large number of drug candidate involves
a hierarchy of computational processes and activites: at the lowest level is
the specific {\bf algorithm} (or equivalentally {\bf protocol}) that is used
to compute the binding free energy of a single drug candidate. There are
multiple protocols that can be used, each comes with its specific trade-offs.
For example, TIES and ESMACS are two protocols to compute binding affinities
that differ in their accuracy but also their computational cost. The
computational instance implementing a protocol -- specific parameter values,
number of simulations and other specific computational aspects of that the
protocol constitute the {\bf workflow}. A given binding affinity calculation
for a given drug candidate can be computed using different workflows. A
workflow may be fully specified a priori, or it may adapt some aspect, say
parameters as a consequence of intermediate results. Typically, there is a
1-to-many relationship between protocol and workflows. When multiple drug
candidates need to be evaluated with certain constraints and a defined
objective The entire computational activity, e.g., computing binding
affinities for multiple drug candidates, which need to be evaluated with
certain constraints and a defined objective is referred to as a {\bf
computational campaign}.

The objective of the computational campaign of relevance to this paper is to
maximize the number of drug candidates for which the binding affinity of each
inidividual candidate is determined to within a (given) acceptable level of
error. The campaign is constrained by the computational resources available
(measured in thousand of core-hours). To meet this objective, each workflow
computing the binding affinity of a drug candidate is adaptively executed.

Executing scalable and adaptive simulation methods directly HPC using
ensemble-based methods presents several
challenges~\cite{cosb18,adaptivebiomolecular}. HTBAC is motivated by these
specific challenges in the context of a computational campaign to compute
ensemble-based free energy methods.


% By designing ensemble-based methods to use MPI, it is difficult to implement
% existing applications in other programming models. This is especially the
% case when running ensemble-based methods on clouds. The ensemble-based
% methods designed using MPI specifically on HPCs cannot be transported without
% rewriting the algorithm to fit into a specific programming model such as
% MapReduce.

% As we reach higher scales of concurrency required by ensemble-based methods,
% the use of middleware becomes a necessity in order to provide reliable
% coordination and distribution mechanisms and low performance overheads. Some
% existing software tools enable execution of ensemble-based methods on HPCs,
% yet only offer end-to-end solutions and lack in providing portability and
% flexibility, bounding the user to a particular infrastructure. In
% domain-specific projects or languages, the user is limited to
% adaptive features that target specific adaptive sampling methods, and require
% explicit management of resources.

% Domain scientists who wish to create adaptive methods require a solution that
% provides the flexibility to create ensemble-based free energy
% calculations with the ability to incorporate user-defined adaptivity. 


% \begin{itemize}
%   \item Binding affinity calculations in general
%   \item Binding affinity using ensemble approach
%   \item How are people doing scalable binding affinity calculations
%   currently
%   \item Overview of adaptive algorithms used for free energy calculations
%   i.e. MSM, REST, EE
%   \item How are adaptive algorithms managed
%   \item What are the existing tools: HPCs: MPI mainly, clouds and
%   ensemble-based middleware
%   \item Appreciate why ensemble-based methods are not done in clouds
% \end{itemize}

% In this approach the Hamiltonian used in the calculation is gradually
% transformed from a description of an initial compound into that of a second
% molecule and the difference in binding strength between the two is
% computed.

% The improvements in hardware, namely GPGPUs, are leveraging more rapid
% simulations, however to deliver clinical insight for drug screening
% campaigns requires more rapid timescales and better binding precision...
% \jdnote{EnTK paper mentions D.E. Shaw paper on specialized hardware and
% running single long simulation}

\begin{itemize}
  \item Binding affinity calculations in general
  \item Binding affinity using ensemble approach
  \item How are people doing scalable binding affinity calculations currently
  \item Overview of adaptive algorithms used for free energy calculations i.e. 
  MSM, REST, EE 
  \item How are adaptive algorithms managed
  \item What are the existing tools: HPCs: MPI mainly, clouds and 
  ensemble-based middleware 
  \item Appreciate why ensemble-based methods are not done in clouds
\end{itemize}

Free-energy calculations using molecular dynamics simulations are now used in 
a wide range of research spanning thermodynamics, protein folding, etc. 
Free-energy calculation require 3 main components, a suitable model Hamiltonian,
sampling protocol and estimator for free energy. Several approaches in 
free-energy binding exist, amongst which relative binding affinity calculations 
are generating accurate predictions delivering considerable promising for
hit-to-lead optimization\jdnote{something like this to get the ball rolling}. 

The improvements in hardware, namely GPGPUs, are leveraging more rapid 
simulations, however to deliver clinical insight for drug screening campaigns 
requires more rapid timescales and better binding precision...
\jdnote{EnTK paper mentions D.E. Shaw paper on specialized hardware and 
running single long simulation}

Ensemble-based simulations, which utilize shorter can deliver at the timescales
and precision necessary to meet the requirements of drug campaings...

In bio-simulations, several classes of ensemble-based algorithms exist to 
compute binding free energy. Each exist to tackle different problem spaces. 
For example, an increasingly popular approach is to
used Markov state models (MSM)to learn a simplified representation of the
phase space explored by simulations and use this to inform which regions are
further sampled in future simulations.

Other adaptive methods such as replica exchange and well-known variations of
replica exchange i.e. replica exchange with solute tempering (REST) use the
Metropolis-Hastings criteria to make periodic decisions that impact which
regions of the phase space are being sampled\jdnote{@dww which citation goes
well with this? the wang2015 paper?}.

Individual simulations that learn by exchanging information have shown better 
sampling results. Adding additional simulations in regions where the curve is 
changing most rapidly reduces errors on the predicated binding affinity. 

Executing simulations directly on HPCs using ensemble-based methods presents 
several challenges. Most users rely on using MD kernels directly with 
execution models such as MPI to run simulations. 

Why is running adaptive ensemble-based simulations on clouds a big faux-pas?

Some existing workflow tools enable scalable execution of ensemble-based 
approaches on HPCs, yet only offer end-to-end solutions...\jdnote{BAC is one 
that is domain specific, we have others}


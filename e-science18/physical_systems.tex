% ---------------------------------------------------------------------------
\subsection{Physical system description}

Scientific and computational improvements require validation across a number
of protein ligand complexes. We selected 4 proteins and 8 ligands or ligand
pairs to run adaptive free energy calculations. The proteins are the Protein
tyrosine phosphatase 1B (PTP1B), the Induced myeloid leukemia cell
differentiation protein (MC1), tyrosine kinase 2 (TYK2) and the
bromodomain-containing protein 4 (BRD4). Four ligands are alchemical
transformations from one to another (used in TIES), four are single ligands
suitable for absolute free energy calculations (used in ESMACS). All systems
were taken from previously published studies~\cite{Bhati2017}.

Simulations were setup using our automated tool, BAC~\cite{Sadiq2008}. This
process includes parametrization of the compounds, solvation of the
complexes, electrostatic neutralization of the systems by adding counterions
and generation of configurations files for the simulations. The AMBER
ff99SB-ILDN~\cite{Lindorff-Larsen2010} force field was used for the proteins,
and TIP3P was used for water molecules. Compound parameters were produced
using the general AMBER force field (GAFF)~\cite{Wang2004} with Gaussian
03~\cite{Frisch} to optimize compound geometries and to determine
electrostatic potentials at the Hartree–Fock level (with 6-31G** basis
functions). The restrained electrostatic potential (RESP) module in the AMBER
package~\cite{Case2005} was used to calculate the partial atomic charges for
the compounds. All systems were solvated in orthorhombic water boxes with a
minimum extension from the protein of 14 \AA\, resulting in systems with
approximately 40,000 atoms.

% \subsection{Target protein: BRD4}
% 
% Bromodomain-containing proteins, and in particular the four members of the
% BET (bromodomain and extra terminal domain) family, are currently a major
% focus of research in the pharmaceutical industry. Small molecule inhibitors
% of these proteins have shown promising preclinical efficacy in pathologies
% ranging from cancer to inflammation. Indeed, several compounds are
% progressing through early stage clinical trials and are showing exciting
% early results~\cite{Theodoulou2016}. One of the most extensively studied
% targets in this family is the first bromodomain of bromodomain-containing
% protein 4 (BRD4-BD1) for which extensive crystallographic and ligand
% binding data are available~\cite{Bamborough2012}.
% 
% We have previously investigated a congeneric series of ligands binding to
% BRD4-BD1 (we shall from now on refer to this are simply BRD4) using both
% ESMACS and TIES. This was performed in the context of a blind test of the
% protocols in collaboration with GlaxoSmithKline~\cite{Wan2017brd4}. The
% goal was to benchmark the ESMACS and TIES protocols in a realistic drug
% discovery scenario. In the original study, we investigated chemical
% structures of 16 ligands based on a single tetrahydroquinoline (THQ)
% scaffold. % ~\cite{Gosmini2014}. Here we focus on the first seven of these
% ligands to test and refine the protocols used and the way in which they
% were executed. The results of our previous work provide a benchmark of both
% accuracy and statistical uncertainty to which we can compare our new
% results.
% 
% Initial coordinates for the protein-ligand system were taken from the X-ray
% crystal structure PDB ID: 4BJX. % ~\cite{Wyce2013}. This structure contains
% a ligand based on the THQ template and other ligands were aligned with this
% common scaffold.

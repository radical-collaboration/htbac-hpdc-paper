\documentclass[conference]{IEEEtran}

\usepackage{amsmath}
\usepackage{amssymb}
\usepackage{array}
\usepackage{booktabs}
\usepackage{color}
\usepackage{float}
\usepackage{graphicx}
\usepackage{ifpdf}
\usepackage[utf8]{inputenc}
\usepackage{keyval}
\usepackage{listings}
\usepackage{moresize}
\usepackage{multirow}
\usepackage[numbers,sort&compress,square]{natbib}
\usepackage{paralist}
\usepackage{rotating}
\usepackage{soul}
% \usepackage[style=base]{subcaption}
\usepackage{srcltx}
\usepackage{url}
\usepackage[dvipsnames]{xcolor}
\usepackage{xspace}
\usepackage{wrapfig}
\usepackage{hyperref}
% \usepackage{caption}
\usepackage{enumitem}

\definecolor{listinggray}{gray}{0.95}
\definecolor{darkgray}{gray}{0.7}
\definecolor{commentgreen}{rgb}{0, 0.4, 0}
\definecolor{darkblue}{rgb}{0, 0, 0.6}
\definecolor{purple}{rgb}{0.6, 0, 0.6}
\definecolor{middleblue}{rgb}{0, 0, 0.75}
\definecolor{darkred}{rgb}{0.4, 0, 0}
\definecolor{brown}{rgb}{0.5, 0.5, 0}
\definecolor{dkgreen}{rgb}{0,0.5,0}
\definecolor{orange}{rgb}{1,.5,0}
\definecolor{dandelion}{cmyk}{0,0.29,0.84,0}

\lstset{ 
  backgroundcolor=\color{white},   % choose the background color; you must add \usepackage{color} or \usepackage{xcolor}; should come as last argument
  basicstyle=\ttfamily\footnotesize,        % the size of the fonts that are used for the code
  breakatwhitespace=false,         % sets if automatic breaks should only happen at whitespace
  breaklines=true,                 % sets automatic line breaking
  captionpos=b,                    % sets the caption-position to bottom
  commentstyle=\color{purple},    % comment style
  deletekeywords={...},            % if you want to delete keywords from the given language
  escapeinside={\%*}{*)},          % if you want to add LaTeX within your code
  extendedchars=true,              % lets you use non-ASCII characters; for 8-bits encodings only, does not work with UTF-8
  % frame=single,                    % adds a frame around the code
  keepspaces=true,                 % keeps spaces in text, useful for keeping indentation of code (possibly needs columns=flexible)
  keywordstyle=\color{orange},       % keyword style
  language=python,                 % the language of the code
  morekeywords={*,...},            % if you want to add more keywords to the set
  numbers=left,                    % where to put the line-numbers; possible values are (none, left, right)
  numbersep=5pt,                   % how far the line-numbers are from the code
  numberstyle=\tiny\color{darkgray}, % the style that is used for the line-numbers
  rulecolor=\color{black},         % if not set, the frame-color may be changed on line-breaks within not-black text (e.g. comments (green here))
  showspaces=false,                % show spaces everywhere adding particular underscores; it overrides 'showstringspaces'
  showstringspaces=false,          % underline spaces within strings only
  showtabs=false,                  % show tabs within strings adding particular underscores
  stepnumber=2,                    % the step between two line-numbers. If it's 1, each line will be numbered
  stringstyle=\color{commentgreen},     % string literal style
  tabsize=2,                       % sets default tabsize to 2 spaces
  % title=\lstname                   % show the filename of files included with \lstinputlisting; also try caption instead of title
}

\usepackage[normalem]{ulem}
\makeatletter
\def\cyanuwave{\bgroup \markoverwith{\lower3.5\p@\hbox{\sixly \textcolor{cyan}{\char58}}}\ULon}
\def\reduwave{\bgroup \markoverwith{\lower3.5\p@\hbox{\sixly \textcolor{red}{\char58}}}\ULon}
\def\blueuwave{\bgroup \markoverwith{\lower3.5\p@\hbox{\sixly \textcolor{blue}{\char58}}}\ULon}
\font\sixly=lasy6 % does not re-load if already loaded, so no memory problem.
\makeatother

\usepackage{pgfplots}
\pgfplotsset{compat=newest}
\usepgfplotslibrary{fillbetween}
\usetikzlibrary{patterns}

\usepackage{siunitx}
\DeclareSIUnit{\calorie}{cal}
\usepackage[outline]{contour}

\usepackage{makecell}

\input{include}

\title{Concurrent and Adaptive Extreme Scale Binding Free Energy
Calculations}

\author{
\IEEEauthorblockN{}
\IEEEauthorblockA{RADICAL Laboratory, Electrical and Computer Engineering,
                    Rutgers University, New Brunswick, NJ, USA}
}

\begin{document}
\maketitle

\begin{abstract}
The efficacy of drug treatments depends on how tightly small molecules bind
to their target proteins. Quantifying the strength of these interactions (the
so called ‘binding affinity’) is a grand challenge of computational
chemistry, surmounting which could revolutionize drug design and provide the
platform for patient specific medicine. Recently, evidence from blind
challenge predictions and retrospective validation studies has suggested that
molecular dynamics (MD) can now achieve useful predictive accuracy ($\leq$ 1
kcal/mol). This accuracy is sufficient to greatly accelerate hit to lead and
lead optimization.

To translate these advances in predictive accuracy so as to impact clinical
decision making requires that binding free energy results must be turned
around on reduced timescales without loss of accuracy. This demands advances
in algorithms, scalable software systems, and intelligent and efficient
utilization of supercomputing resources.

We introduce the use of a framework called HTBAC, designed to support the
aforementioned requirements of accurate and scalable drug binding affinity
calculations, to marshal huge simulation campaigns. HTBAC facilitates the
execution of the numbers of simulations while supporting the adaptive
execution of algorithms to support runtime decisions during execution to 
increase fidelity of calculations and increase the rate of converge.
\end{abstract}


% ---------------------------------------------------------------------------
% INTRODUCTION
% ---------------------------------------------------------------------------
\section{Introduction}
\label{sec:intro}
\textbf{Start with importance of binding free energy for drug design}

Binding free energy calculations rely on the physics-based molecular simulations 
and statistical mechanisms in order to predict structure-based drug design. 
Based on recent improvements in the accuracy of binding free energy, free energy 
calculations are now becoming reliable methods in the drug discovery pipeline. 
These improvements can be attributed to many advances in methodologies, as well 
as enhancements in hardware. Specifically, ensemble-based 
binding free energy calculations, which favor multiple shorter simulation 
trajectories over few longer simulations are showing improvements in biological 
time scales and can now leverage binding affinity calculations to provide 
meaningful clinical insight into computational drug campaigns and potentially 
reduce the cost of expensive experimental screening programs. 
These approaches could also be used to determine the influence of individual 
patients genetic sequence on drug efficacy in the context of clinical decision 
support. In order for these approaches to gain traction, requires using 
simulation protocols which have well-defined uncertainty and consistently 
produce statistically meaningful results. 

\textbf{2 prominent free energy protocols} 

Computational drug campaigns rely on rapid screening of thousands of compounds. 
At the start of a drug design campaign an initial screening of candidate 
compounds is conducted to filter out the worst binders before using more 
sensitive methods to refine the structure. Two prominent ensemble-based 
free energy protocols, ESMACS and TIES~\cite{Bhati2017} have demonstrated the 
ability to filter and refine the drug design process. The ESMACS (enhanced 
sampling of molecular dynamics with approximation of continuum solvent) protocol 
provides an ``approximate'' endpoint method used to screen out poor binders. The 
TIES protocol (thermodynamic integration with enhanced sampling) uses a more 
rigorous ``alchemical" thermodynamic integration approach. These protocols have 
demonstrated statistically meaningful results at timescales relevant for 
industrial computational drug campaign. 

\textbf{Scalable computing approaches that try to utilize CPU hours and cores}

Scalable simulation computing approaches such as ensemble-based free energy 
protocols in drug campaigns are utilizing the growing number of compute 
resources by scaling the number of highly parallel simulations. Most of these 
computational approaches focus on the utilization of core hours using 
fire-and-forget executions, but not many try to use core-hours effectively. 

\textbf{...But not many approaches try to use core-hours effectively}

To address the rapidly growing screening process of production scale drug 
campaigns which require screening of millions of compounds, tools that only 
leverage the scalable computation of varied binding free energy calculations on 
HPC resources are not enough to deliver on timescales that are clinically 
warranted. 

\textbf{In order to do so requires advances in methods and resource efficiency}

Moreover, free energy binding protocols need better sophistication to 
maximize resources for simulations that yield the best accuracy and precision 
given set of resources or fixed computing time. However, computational drug 
campaigns screen for compounds that are unique in chemical properties. The main 
challenge lies in the variability of statistical uncertainty of the free energy 
binding between compounds. Ensemble-based free energy protocols leverage 
generality of chemical properties to maintain convergence of binding free energy 
across compounds with variable degrees of freedom, but at the cost of 
inefficiency in resources consumption. Moreover, generality also means that 
once accuracy thresholds are reached, the protocol has converged, which prevents
protocols from achieving better accuracy for certain compounds. \jdnote{needs a 
stronger motivation} One of the 
difficulties involved in defining adequate protocols is that differences in 
drug candidate chemistry (or protein sequence) can alter simulation behavior and 
the convergence of binding strength estimates. For example, not all changes 
induced in protein shape or behavior are local to the drug binding site and, in 
some cases, simulation protocols will need to adjust to account for complex 
interactions between drugs and their targets within individual studies. Advances 
in methodologies can provide the ability to capture complexities in simulations 
for a given compound, however these methodologies rely on learning the positions 
of simulations during execution.\jdnote{alternatively we could include this 
instead of the last sentence of this paragraph: A more strategic concern is that 
during a study the requirements may change.} 
 

\textbf{Specific adaptive methods within binding free energy can enhance 
clinical insight}

Here we show how adaptive approaches within ensemble-based free energy 
protocols are designed to capture unique chemical properties and thereby 
customize the simulations for a candidate in a way that makes the most 
effective use of computational resources in order to deliver better fidelity of 
statistical uncertainties than general approaches. For drug campaigns, we can 
leverage adaptive approaches to support drug campaigns for clinical insight. 

\textbf{Suggest solution to address execution of scalable adaptive methods}

\jdnote{following paragraphs are taken from EnTK, will fix}
Most tools do not support the ability to encode and execute more complex 
algorithm logic, let alone provision runtime capabilities that can change 
resource configurations based on intermediately generated data.  
We need a tool that can leverage scalable execution of adaptive algorithms.

Implementation of adaptive algorithms on high performance clusters are 
predicated on adaptive runtime systems that provide the ability to make 
runtime decisions based on intermediate results and can manage resources 
efficiently. To achieve scalability and efficiency, such adaptivity cannot be 
performed via user intervention and hence automation of the control logic and 
execution is important. HTBAC enables the scalable execution of adaptive 
algorithms. 


% Leveraging ensemble-based simulations, we can now explore new adaptive sampling 
% schemes can improve sampling quality of ligand binding affinity of individual drug candidates

% \textbf{Different adaptivity schemas exist, highlight intra-protocol adaptivity but other forms of adaptivity exist}

This paper makes the following contributions:
\begin{itemize}
  \item Identifies the challenge in advancing adaptive algorithms and 
  methodologies
  \item Shows the importance of using adaptive approaches within ensemble-based
  free energy protocols to improve binding affinity accuracy given a fixed 
  amount of computing
  \item Provides the adaptive software solution (HTBAC) that enables the 
  scalable execution of adaptive algorithms
  \item Demonstrates the capability to execute adaptive applications at scale 
  and validates the scientific results from these applications.
\end{itemize}

\textbf{results of scaling experiments and intra-protocol adaptivity experiments}

This paper is organized as follows: 
Section~\ref{sec:science-drivers} describes ESMACS and TIES as ensemble-based 
free energy protocols as well as an adaptive methodology that is implemented 
within the protocol that motivates the need for higher precision given limited 
resources. 
Section~\ref{sec:related-work} describes the motivation for ensemble-based 
approaches and existing solutions and their limitations in their ability to 
support adaptive methods.  
Section~\ref{sec:htbac} describes the design and implementation of 
HTBAC--the software tool that addresses the requirements of scalable, 
adaptive methods. 
In Section~\ref{sec:experiments} we demonstrate experiments of scalability, 
validation of results, and adaptive simulation methods. We show that given a 
fixed amount of computing resources, we can achieve better accuracy and hence 
better time to solution using adaptive simulation methods. 








% ---------------------------------------------------------------------------
% S2
% ---------------------------------------------------------------------------
\section{Related Work}
\label{sec:related-work}
The strength of ligand binding is determined by a thermodynamic property known
as the binding free energy (or binding affinity). 
One promising technology for
estimating binding free energies and the influence of protein and ligand
composition upon them is molecular dynamics (MD)~\cite{Karplus2005}. 
A diversity of methodologies have been developed to calculate binding affinities
MD sampling~\cite{Mobley2012} and blind tests show that many have considerable
predictive potential~\cite{Mey2017, Yin2017}.
The development of commercial approaches that claim accuracy of below 1 
kcal mol$^{-1}$
% ~\cite{Wang2015} 
has led to increased interest from the 
pharmaceutical industry~\cite{Ganesan2017}.
The same technologies have also shown promise in analysing the influence of 
protein mutations on drug binding~\cite{Mondal2016, Bunney2015}.
These developments have also spurred renewed interest in the factors affecting 
the performance of different free energy 
methods~\cite{Aldeghi2017, Cappel2016, Ruiter2016}.
As these developments combine with increased computational power it is becoming 
possible to integrate the information gained from large numbers of simulations 
to provide more predictive models using machine learning 
approaches~\cite{Ash2017}. It is this approach of combining predictive modeling 
with integrative technologies which provides one of the most promising 
avenues for developing decision support tools for drug discovery and 
clinical support.

As binding free energy approaches based on MD become more widely used, 
particularly by non-specialists, automated tools for system and simulation 
preparation have been developed~\cite{Gapsys2015, Doerr2016, Rizzi}, including 
our own tool the binding affinity calculator (BAC)~\cite{Sadiq2008}.
Using BAC, we have designed two protocols
with the demands of clinical decision support and drug design applications in
mind: ESMACS (enhanced sampling of molecular dynamics with approximation of
continuum solvent)~\cite{Wan2017brd4} and TIES (thermodynamic integration with
enhanced sampling)~\cite{Bhati2017}. 
The former protocol is based on variants
of the molecular mechanics Poisson-Boltzmann surface area (MMPBSA), which is
an ``approximate'' end-point method~\cite{Massova1999}. 
The latter on the rigorous ``alchemical'' thermodynamic integration (TI)
approach~\cite{Straatsma1988}. 

Using these protocols, we have demonstrated the lack of reproducibility of
individual simulations for a variety of protein systems, with calculations for
the same protein-ligand combination using almost identical initial conditions
producing widely varying results (binding affinities varying by up to 12 kcal
mol $^{-1}$ for small molecules, whilst flexible ligands can vary even
more)~\cite{Sadiq2010, Wright2014}. Indeed, our work has revealed how
completely unreliable single simulation based approaches are. We have
shown that averaging across multiple runs reliably produces results in
agreement with previously published experimental findings~\cite{Sadiq2010,
Wan2011, Wright2014, Bhati2017, Wan2017brd4, Wan2017trk}, and correctly
predicted the results of experimental studies performed by colleagues in
collaboration~\cite{Bunney2015}. We term this approach ensemble molecular
dynamics, ``ensemble'' here referring to the set of individual (replica)
simulations conducted for the same physical system. 

Both TIES and ESMACS protocols are ensemble based. 
Basing these computations on the direct calculation of ensemble averages 
facilitates the determination of statistically meaningful results along with 
complete control of errors. 
In addition to accuracy and
precision, in order for MD simulations to attain widespread use in industrial
(and potentially clincal) settings, it is necessary that results can be
obtained soon enough in order that the results can feed into decision making
processes. 
These motivations provided the impetus behind our automated model building and 
simulation input generation package, the binding affinity calculator 
(BAC)~\cite{Sadiq2008}. As we scale up to larger datasets the use of the ensemble
approaches will increasingly necessitate the use of middleware to provide
reliable coordination and distribution mechanisms with low performance
overheads.
These considerations provide the motivation behind the development of HTBAC.
%The necessity for rapid turn around times places additional demands
%on simulation protocols which need to be optimized to gain results with a
%short turn around time. Further to the scientific and practical considerations
%outlined above, it is vital that reliable uncertainty estimates are provided
%alongside all quantitative results for simulations to provide actionable
%predictions.

%We have developed a number of free energy calculation protocols based on the
%use of ensemble molecular dynamics simulations with the aim of meeting these
%requirements~\cite{Sadiq2008, Sadiq2010, Wan2017brd4, Wan2017trk}. 




% Atomistically detailed models of the drug and target protein can be used as
% the starting point for MD simulations to study the influence of changes in
% either drug or protein composition on drug binding. The chemistry of the
% system is encoded in what is known as a potential ~\cite{Karplus2002}. In the
% parameterization of the models, each atom is assigned a mass and a charge,
% with the chemical bonds between them modeled as springs with varying
% stiffness. Newtonian mechanics is then used to follow the dynamics of the
% protein and drug and, using the principles of statistical mechanics, estimates
% of thermodynamic properties obtained from simulations of single particles.

% The potentials used in the simulations are completely under the control of the
% user. This allows the user to manipulate the system in ways which would not be
% possible in experiments. A particularly powerful example of this are the so
% called ``alchemical'' simulations in which the potential used in the
% simulation changes, from representing a particular starting system into one
% describing a related target system during execution. This allows for the
% calculation of free energy differences between the two systems, such as those
% induced by a protein mutation.

% MD simulations can reveal how interactions change as a result of chemical
% alterations, and account for the molecular basis of drug efficacy. This
% understanding can form the basis for structure-based drug design as well as
% helping to target existing therapies based on protein sequence. However,
% correctly capturing the system behavior poses at least two major challenges:
% The approximations made in the potential must be accurate enough
% representations of the real system chemistry; and sufficient sampling of phase
% space is also required. Our previous work demonstrated \cite{Sadiq2010,
% Wan2011} that running multiple MD simulations based on the same system and
% varying only in initial velocities offers a highly efficient method of
% obtaining accurate and reproducible estimates of the binding affinity. While
% the accuracy of force fields could be a source of error, we know from our work
% to date that the very large fluctuations in trajectory-based calculations
% account for the lion’s share of the variance (hence also uncertainty) of the
% results.





% ---------------------------------------------------------------------------
% S3
% ---------------------------------------------------------------------------
\section{Science Drivers}
\label{sec:science-drivers}
In this section we provide details about ESMACS and TIES specifications and
about adaptive methodologies using TIES\@. We conclude with a description and
validation of the physical systems used in this work.

% ---------------------------------------------------------------------------
\subsection{ESMACS and TIES}\label{ssec:esm_ties}

ESMACS and TIES~\cite{Wan2017brd4, Bhati2017} are two free energy calculation
protocols that implement absolute and relative methods, respectively.
Absolute free energy methods calculate the binding affinity of a
\emph{single} drug molecule to a protein, while relative methods calculate
the \emph{difference} in binding affinity between two (usually similar in
structure) drug molecules. Both protocols are designed to use an ensemble MD
simulation approach to enhance the reproducibility and accuracy of standard
free energy calculation techniques (MMPBSA~\cite{Massova1999} in the case of
ESMACS and thermodynamic integration~\cite{Straatsma1988, Straatsma1991} in
TIES). The use of ensemble averaging allows tight control of error bounds in
the resulting free energy estimates.

ESMACS and TIES consists of three main steps: minimization, equilibration and
production MD (in its current implementation all MD steps are conducted in
NAMD~\cite{Phillips2005}). In practice, the equilibration phase is broken
into multiple steps to ensure that the size of the simulation box does not
alter too much over the simulation. During these steps, positional
constraints are gradually released from the structure and the physical system
is heated to a physiologically realistic temperature.

Whilst both protocols share a common sequence of steps, the make-up of the
ensemble is different. In ESMACS, an ensemble consists of a set of 25
\textbf{replicas}, i.e., identical simulations differing only in the initial
velocities assigned to each atom. In TIES, the ensemble contains a set of
\textbf{$\lambda$} windows, each spawning a set of replicas. As a
transformation parameter $\lambda$ increases from 0 to 1, the system
description transforms from containing an initial drug to a target compound
via a series of hybrid states. Sampling along $\lambda$ is then required to
compute the difference in binding free energy. In previous studies, TIES has
been deployed using 65 replicas, evenly distributed among 13 $\lambda$
windows. Following the completion of the simulation steps, both protocols
require the execution of free energy analysis steps. The detailed composition
of ESMACS and TIES protocols is shown in
Fig.~\ref{fig:ties_esmacs_application}.

\begin{figure}
  \centering
  \includegraphics[width=\columnwidth]{figures/ties_esmacs_application_model.pdf}
  \caption{TIES and ESMACS protocols consist of simulations steps followed by
  analysis step(s). ESMACS contains 25 replicas per simulation step; TIES
  contains 5 replicas per $\lambda$ window. We model TIES with 13 $\lambda$
  windows, spawning 65 replicas in each simulation step. All replicas
  simulate 6ns.}\label{fig:ties_esmacs_application}
\up{}
\up{}
\end{figure}

% ---------------------------------------------------------------------------
\subsection{The Value of Adaptivity}\label{ssec:adapt_ties}

The main driver for adaptivity is that computational campaigns will typically
involve compounds with a wide range of chemical properties which can impact
the time to convergence and the type of sampling required to gain accurate
results. There may be cases where it is important to increase the sampling of
phase space, possibly through expanding the ensemble. In general, there is no
way to know exactly which calculation setup a particular system requires
before runtime.

Another driver of adaptivity is that, on occasion, alchemical methods may
converge very slowly. This means that the most effective way to gain accurate
and precise free energy results on industrially or clinically relevant
timescales is to adapt either the workflow corresponding to a specific
protocol or adapt different workflows in relation to each other. The latter is
referred to as \textbf{inter-protocol} adaptivity; the former as
\textbf{intra-protocol} wherein, for example, the parameter values associated
with a specific protocol might change. With thousands of workflows
(corresponding to a protocol instances) to adapt in different ways, this has
the potential to allow for significant optimization.

% simulations, often employing multiple analysis methodologies, this provides
% the most effective way to utilize these techniques and resources at scale.

In TIES, the change in free energy associated with the transformation is
calculated using an adaptive quadrature function which numerically integrates
the values of the $\partial U/\partial\lambda$ across the full set of
simulated $\lambda$ windows. Obtaining accurate and precise results from TIES
using adaptive quadratures requires that the $\lambda$ windows correctly
capture the changes of $\partial U/\partial\lambda$ over the transformation.
This behavior is highly sensitive to the chemical details of the compounds
being studied and varies considerably among candidates. Typically, $\lambda$
windows are evenly spaced between 0 and 1 with the spacing between them set
before execution at a distance determined by the simulator to be sufficient
for a wide range of systems.

However, the number or the location of the $\lambda$ windows that will most
impact the calculation are not known \textit{a priori}, and varies across
candidates. As each window requires multiple simulations, sampling with a
high frequency is expensive. Approximations using evenly spaced $\lambda$
windows reach an acceptable accuracy threshold but adaptive placement of
$\lambda$ windows is likely to better capture the shape of the $\partial
U/\partial\lambda$ curve, leading to more accurate and precise results for a
comparable computational cost.

% we use adaptive middleware to automate runtime decisions based
% on partial simulation data, and redistribute resources at runtime to support
% dynamically generated simulations. We

In this work, we focus on intra-protocol adaptivity
which relies on intermediate runtime results \textit{within} a protocol
instance to define the following set of simulations. Instead of approximating
the placement of all the $\lambda$ windows prior to execution, we run TIES
with less $\lambda$ windows and shorter bursts of simulations, analyzing
intermediate runtime results (i.e., trajectories) to seed new and ideally
placed $\lambda$ windows.
\subsection{Physical system description}

The scientific and or computational improvements require validation across a number of protein ligand complexes. We selected 4 proteins and 8 ligands or ligand pairs to run the adaptive relative free energy calculations. The systems were previously used for non-adaptive simulations by \cite{} et. al., offering a baseline comparison against our results. 

The selected proteins are the Protein tyrosine phosphatase 1B (PTP1B), the Induced myeloid leukemia cell differentiation protein (MC1), tyrosine kinase 2 (TYK2) and the bromodomain-containing protein 4 (BRD4). The ligand complexed with the protein fall into two categories. Four of the eight ligands are alchemical transformations from one to another (used in the TIES workflow), the rest being single ligands suitable for absolute free energy calculations, in our case for ESMACS.

Preparation and setup of the simulations were implemented using our automated tool, BAC~\cite{Sadiq2008}. This process includes parametrization of the compounds, solvation of the complexes, electrostatic neutralization of the systems by adding counterions and generation of configurations files for the simulations. The AMBER ff99SB-ILDN \cite{Lindorff-Larsen2010} force field was used for the proteins, and TIP3P was used for water molecules. Compound parameters were produced using the general AMBER force field (GAFF)~\cite{Wang2004} with Gaussian 03~\cite{Frisch} to optimize compound geometries and to determine electrostatic potentials at the Hartree–Fock level with 6-31G** basis functions. The restrained electrostatic potential (RESP) module in the AMBER package~\cite{Case2005} was used to calculate the partial atomic charges for the compounds. All systems were solvated in orthorhombic water boxes with a minimum extension from the protein of 14 \AA\ (the resulting systems contain approximately 40 thousand atoms).

% \subsection{Target protein: BRD4}
% 
% Bromodomain-containing proteins, and in particular the four members of the BET
% (bromodomain and extra terminal domain) family, are currently a major focus of
% research in the pharmaceutical industry. Small molecule inhibitors of these
% proteins have shown promising preclinical efficacy in pathologies ranging from
% cancer to inflammation. Indeed, several compounds are progressing through
% early stage clinical trials and are showing exciting early
% results~\cite{Theodoulou2016}. One of the most extensively studied targets in
% this family is the first bromodomain of bromodomain-containing protein 4
% (BRD4-BD1) for which extensive crystallographic and ligand binding data are
% available~\cite{Bamborough2012}.
% 
% We have previously investigated a congeneric series of ligands binding to
% BRD4-BD1 (we shall from now on refer to this are simply BRD4) using both
% ESMACS and TIES. This was performed in the context of a blind test of the
% protocols in collaboration with GlaxoSmithKline~\cite{Wan2017brd4}. The goal
% was to benchmark the ESMACS and TIES protocols in a realistic drug discovery
% scenario. In the original study, we investigated chemical structures of 16
% ligands based on a single tetrahydroquinoline (THQ) scaffold.
% % ~\cite{Gosmini2014}. 
% Here we focus on the first seven of these ligands to test
% and refine the protocols used and the way in which they were executed. The
% results of our previous work provide a benchmark of both accuracy and
% statistical uncertainty to which we can compare our new results.
% 
% Initial coordinates for the protein-ligand system were taken from the X-ray
% crystal structure PDB ID: 4BJX.
% % ~\cite{Wyce2013}. 
% This structure contains a
% ligand based on the THQ template and other ligands were aligned with this
% common scaffold. 



% ---------------------------------------------------------------------------
% S4
% ---------------------------------------------------------------------------
\section{HTBAC: Design and Implementation}
\label{sec:htbac}
% ---------------------------------------------------------------------------
\subsection{Computational Challenges}


\jhanote{we need to elevate the challenge above (i) ensembles, (ii) and
separate adaptive from non-adaptive challenges}


Most advances in high-performance computing have focused on the scale,
performance and optimization of single long simulations. However, due to the
end of Dennard scaling and methodological advances, many applications are now
formulated using multiple, shorter ensemble-based simulations. Yet, there are
limited software solutions that can execute scalable heterogeneous
computational tasks. HTBAC builds upon middleware solutions that overcome
these barriers...

% To encode an algorithm into a set of tasks requires a workflow system that
% exposes an interface to describe tasks, dependencies, and an execution
% pattern. Once the algorithm is encoded into a set of tasks, the workflow
% system is required to translate tasks into executables, coordinate task
% dependencies, and acquire and manage resources.

%\jhanote{I'm changing the order of paragraphs: promoting adaptivity, as we
%seem to be building the case for decision making.}

% We define \textbf{adaptivity} as the ability to revisit a prior execution
% decision based on a runtime evaluation. Adapting workflows at runtime requires
% incorporating additional parallelism, redistributing of resources across
% tasks, and analyzing the performance and functionality of added adaptivity to
% evaluate its benefits and further design.

In principle, many workflow management systems can be used to express
ensemble-based workflows, however the

several limitations. Among these, two of the most relevant are long time to
adoption and lack of dynamic capabilities.

% monolithic designs provide end-to-end capabilities to execute workflows on
% heterogeneous and distributed cyberinfrastructure. Other workflow systems
% entail

Furthermore, statistically meaningful results derived from simulations are of
critical interest, as they can leverage additional sampling of simulations
which could lead to an improvement in precision of free energy calculations.
Such decisions typically, cannot be made {\it a priori} and require postmortem
user intervention in order to analyze and spawn additional simulations.


Workflow system with featureful but complex interface models impose
substantial overhead when integrating application workflows in the runtime
system, preventing users from quick and flexible applications prototyping.
Moreover, any intervention during runtime such as spawning additional
simulations could require a redistribution of resources. Current workflow
management systems do not provide adequate support for dynamic resource
utilization, limiting the possibility to modify workflow models at runtime.



% ---------------------------------------------------------------------------
\subsection{HTBAC}

In response to aforementioned challenges and requirements, and the absence of
middleware systems that provide scalable and extensible solutions for the
computation of binding free energy system we designed HTBAC

In this work we validate the generality of HTBAC by implementing the TIES
protocol and investigating its performance with multiple TIES protocol. We
comapare the overheads of HTBAC and demonstrate that its overheads and
performance are invariant of the specific protocol.

HTBAC derives many of the advantages of a lightweight, flexible domain
specific workflow layer from its use of RADICAL-Cybertools (RCT) which are
functionally well-defined and delineated building blocks.
RCT~\cite{review_bb_2016} are engineered to scale across diverse computing
platforms with the rapid ability to design a multitude of workflows. Although
enscapulated from end-users of HTBAC, the two primary RCT components that
HTBAC depends upon are the Ensemble Toolkit (EnTK) and RADICAL-Pilot (RP).

HTBAC promotes binding affinity protocols as the user-facing constructs, and
provides programming abstractions -- Pipeline, Stage and Tasks programming
model (that are consistent with the EnTK) to enable the user to express a
variety of protocols.

HTBAC not only simplifies the expression of complex binding affinity protocols, 
but also provides hitherto unavailable capabilities, viz., the adaptive
forumation/expression and execution of these protocols, without any additional 
programming burden.

In summary, HTBAC allows the simple expression and concurrent execution of
multiple distinct protocols, thereby enabling concurrent screening of drug
candidates.

This suggests three requirements for HTBAC\@: (1) enabling the scalable
execution of concurrent multiple protocols, (2) abstract the complexity of
building protocols, execution and resource management from the user; and (3)
providing adaptive features (i.e., modifiying the task-graph during runtime,
based on previous tasks), to enable more effective protocols, without explicit
resource management requirements.

% Efficient execution of binding affinity calculation protocols on different
% platforms poses several technical challenges. 

	In our earlier work~\cite{dakka2017} we demonstrated how we applied the
PST model to implement the ESMACS protocol into an EnTK application. Here we
expand upon this work to further abstract the protocol implementation details
and resource information from the user. 


% ---------------------------------------------------------------------------
\subsection{Implementation}

In section \ref{sec:science-drivers}, we described two examples of ensemble-
based protocols for computing binding affinities. Each protocol contains
multiple stages with simulations and analysis tasks interspersed, in the most
general case.

% Steps vary in data input dependencies, computational resource requirements and
% MD execution engines. 

% Subsets of these tasks are defined as \textbf{workloads}, i.e., tasks whose
% dependencies have been satisfied at a particular point in time and may be
% executed concurrently. Each protocols' ensemble requires an identical workflow
% with input parameter modifications.

We define each stage as a computational \textbf{task} and the ordered
aggregation of these stages alongside their dependencies as a
\textbf{pipeline}.

HTBAC is implemented in Python, and uses RCT to provide ensemble-execution
capabilities and a runtime system to execute tasks. Ensemble Toolkit (EnTK),
the topmost layer of RCT simplifies the process of creating ensemble-based
applications with complex coordination and communication requirements. The
EnTK API exposes the PST model which consists of three components:
\textbf{Pipeline}, \textbf{Stage}, and \textbf{Task}. HTBAC uses the PST model
to encode \textbf{Protocols},

Consistent with EnTK's programming model, HTBAC also uses the PST model to to
express  protocols.

A workflow is comprised of N$$_P$$ instances of the $$P^th$$ protocol; our
work so far entails a maximum value of P = 2, but in future work P will be
greater > 2.

Once the workflow is described, it is submitted to the \b{Application
Manager} which sets up multiple processes, threads and a RabbitMQ message
queue for communication. EnTK identifies tasks which have dependencies
satisfied and can be executed concurrently. EnTK's \textbf{Execution Manager}
uses the underlying runtime system, RADICAL-Pilot to execute the tasks on
specific target resources.

\begin{figure}
  \centering
   \includegraphics[width=\columnwidth]{figures/isc_htbac_integration_with_entk_RP.pdf}
  \caption{Mapping of flow between HTBAC, EnTK, and RP. The HTBAC API exposes the Protocol
  and Runner components. EnTK serves as the workflow execution system and
  by managing the workflow and workload. RADICAl-Pilot serves as the runtime system}
\label{fig:integration}
\end{figure}


HTBAC exposes two components \textbf{Runner Handle} and \textbf{Protocol}.
The \textbf{Runner} component is a container \jhanote{I don't know what a
container is in this context?} that implements the ESMACS and TIES into the
EnTK PST model abstraction. 

The \textbf{Protocol} class can be intantiated as one of two protocols.
Currently the HTBAC provides the two protocol objects: ESMACS and TIES. For
TIES, each protocol typically corresponds to a single physical system, and
multiple instances can be supported. The TIES protocol object enables the user
to select a physical system, number of replicas, core allocation per replica,
and lambda windows (for alchemical pertubations). Similarly, the ESMAC
protocol enables the user to select ...

% protocol, a user can scale protocol instances to study as many physical systems as
% desired. 

% The specification of protocols and their parameters are passed by the user to
% the \textbf{Runner Handle} which translates the request to EnTK.

	In Section \ref{sec:related-work} we highlighted how an ensemble
simulation approach can both aid sampling and provide improve uncertainty
quantification for free energy calculations. Despite these crucial advantages,
it remains non-trivial for field researchers to write biosimulation
applications that involve individual protocols supporting multiple replicas,
and by extension multiple protocols. With HTBAC, this burden is minimized by
specifying the number of concurrent instances of the \textbf{Protocol} object.
Moreover, the ability to generate multiple protocol instances enables the user
to investigate a range of physical systems (i.e., drug candidates)
concurrently.

%\jhanote{Is this specific to the TIES protocol or to the protocol class. Needs
%clarification.} 
%The same approach has been expanded to facilitate the creation of 
%sub-ensembles where the simulation configuration is altered programatically.
%An example of this is the implementation ofthe TIES protocol,
%where the user controls the $\lambda$ parameter values used in simulatios to 
%control which hybrid system states are sampled.

%A parameter, $\lambda \in [0, 1]$ is set for values between extremes and a
%simulation has to be run for every $\lambda$ value. The values form a function
%of energy and are integrated to obtain the desired results, the
%\emph{relative} binding free energy.


%\subsubsection{System}

%Systems This allows for multiple systems to be tested in the same
%\emph{single} run. A common scenario is the calculations of the binding
%affinity of a set of ligands with the same protein. System itself is just a
%collection of file paths pointing to descriptions of the system, like the
%system structure, topology etc. This class also provides the core/node
%requirments per single run, and reads some of the system descriptions from
%files to fill in the configuration settings.



%The Pipeline-Stage-Task (PST) framework developed by the Radical team
%(cite), and the Ensemble Toolkit (EnTK) built on top of it, offers a
%flexible way to express the molecular dynamics simulation workflows present
%in academia (cite) in terms of the radical pilot execution environment. Here
%we present a proposed mapping between the two (the PST and the MD layers)
%that is both simulation engine and protocol agnostic and allows for the
%compact expression of ensembles frequently used in binding affinity
%calculations.

%\subsection{Overview}

%The framework, called High Throughput Binding Affinity Calculator (HT-BAC),
%a python library, is made up of the following components: Workflow, Step,
%Ensembles and Simulation. These four object are all that is neccessary to
%describe the complex binding affinity caluculations in a generic way.

%\subsection{Workflow}

%The highest level abstraction is the Workflow. It is a container for the
%sequential units that are the simulation steps themselfs, and also contains
%meta-information about the job, like the resource description that the job
%will be running on, the total number of cores (nodes) required to fullfil
%the needs of the simulations and profiling mechanims to measure execution
%time.


% ---------------------------------------------------------------------------
%\jhanote{I don't these "description" should be subsections. Consider 
%"\paragraph{}"}

%\subsection{Step}

%\jhanote{what is a step? It is unclear to the reader. Is "step"  construct
%within HTBAC or is this just a description of the pipeline?}

%The workflow containts an ordered list of \emph{steps}. Steps give
%\jhanote{order?} orderd to the basic building blocks of binding affinity
%calculations. Usually they are (i) minimization (some form of local
%optimization of atom coordinates), (ii) heating, (iii) equilibration and (iv)
%production run.

%Additionally there is one or more steps of analysis at the end. The key
%point, is that these steps \emph{have} to be run consecutively, as they are
%dependent on the previous one. This is ensured by the \texttt{Stage} objects
%of EnTK\@. Each step has list of \texttt{Ensemble}s and a \texttt{Simulation}
%object.

% ---------------------------------------------------------------------------
%\subsection{Ensembles} 

%\jhanote{I'm not sure what we are trying to do here. We've just described
%the abstract/constructs in HTBAC to be protocols. Why are we regressing to
%presenting Ensembles as a construct in HTBAC, when they are a construct of
%EnTK?}

%Ensembles in HTBAC are a powerful construct that allows for extreme
%generalizations. At their core they are a nondestructively multi (forward)
%traversable \emph{iteratable}s. The underlying iterator yields a function that
%modifies a \texttt{Simulation} object to reflect the current state of the
%ensemble. This is similar to the iteratee construct, the only difference being
%that the function is applied to the \emph{same} data consecutively (as opossed
%to chunks of a stream of data).

%The Simulations are then generated for every Step based on what ensembles are
%attached to it. This is done by taking the \texttt{product} of all the
%ensembles, meaning that we go over every possible combination of ensemble
%states, and create a simulation from it. This is equivalent to an $n$ level
%deep for-loop, where $n$ is not known beforehand.

% ---------------------------------------------------------------------------
%\subsection{Simulation}

%Simulation is the lowest level building block. \jhanote{The use of building
%block for simulation will confuse the reader as it is overloading the term.
%Find alternative description.} It maps to the \texttt{Task} object in the PST
%model, and deals with executing the simulation engine, collecting input for
%it, and modifing the configuration of the simulation to reflect the ensembles
%that it is in.

%The main driver of a simulation is the configuration file. This file
%containts the specific settings of the Molecular Dynamics simulation,
%including the thermostat, barostat, constraints, long range force calculation
%methods and more. Some of the settings are general and apply to every
%simulation of a given step, but some are specific to the Ensemble that the
%simulation is in. After the ensemble modifies properties of the Simulation,
%these get propagated and written to the configuration file to be read by the
%executable.

% ---------------------------------------------------------------------------
%\subsection{Sandboxing}

%Sandboxing mechanims are built into the Radical stack for every layer. Each
%separate run has it's own sanbox, and inside each run, all the Tasks have
%their own sandbox too. This means that there is no unintented interference
%between Tasks, and data can be confidently analyised inside each sandbox.
%This mechanism also simplyfies the input data referencing, as the input in
%\emph{guaranteed} to be in the sandbox, and can be referenced relative to it.
%We recommend this sandboxing system for all simulations.

% ---------------------------------------------------------------------------
%\subsubsection{Data flow}

%while sandboxing offers a powerful way to separate the simulations, by the
%inherent nature of workflows, output data from one simulation is required as
%input for the next one. Simulation objects can find their previous
%couterparts by way of the ensemble that they are in. Then data is transfered
%via \textbf{copy-on-demand}, i.e. data is copied \emph{only} if it is known
%to be edited during execution, otherwise only a symbolic link is created to
%point to the original location of the file. This drasticly reduces the copy
%overhead, while still keeping the sandboxes separate.

% ---------------------------------------------------------------------------
\subsection{Adaptability}

%Once we tackle the barrier between the local workflow creation and the remote
%execution, new features become availble, and readily usable by scientists.
%Intraprotocol adaptability is one such new feature.

% ---------------------------------------------------------------------------
\subsubsection{Intraprotocol adaptability}

%while conceptually simple, tradiational execution patters used in academia
%makes this very hard. In HTBAC variables like replica size, specific lambda
%windows or simulatable system are settable on demand, the execution of which
%is automatically handeled by the library. To illustrate: a common scenario is
%the non adequate convergence of the statistical results after running a given
%number of replicas. In HTBAC the replica number can be changed, rerun and the
%results reevaluated. Additionally, logic can be written, to dynamically add
%more replicas until a given convergence tolerance has been reached.



% ---------------------------------------------------------------------------
% S5
% ---------------------------------------------------------------------------
\section{Implementing ESMACS and TIES in HTBAC}
\label{sec:implementation_htbac}
In \S\ref{ssec:esm_ties} we define the structure of the ESMACS and TIES
protocols. Here we provide skeletons of the TIES protocol implemented in
HTBAC\@. In L.~\ref{lst:ties.py} we show a customization of a production
MD simulation step.

\lstinputlisting[language=Python, label={lst:ties.py}, caption={TIES protocol
implemented with HTBAC. We import the predefined protocol `TIES'. We assign
the physical system to the protocol, we instantiate a simulation, customize
its steps (\texttt{replica}, \texttt{lambda}) and assign it to the TIES's
\texttt{step0}. We instantiate a Runner with a resource request and pass the
protocol to it.}]{ties.py}

In \S\ref{ssec:adaptive_execution} we show HTBAC's adaptive execution
capabilities. In L.~\ref{lst:ties_adaptivity.py} we provide an
intra-protocol adaptive implementation of TIES, based on the use-case
of \S\ref{ssec:adapt_ties}.

\lstinputlisting[language=Python, label={lst:ties_adaptivity.py},
caption={Adaptive TIES protocol implemented with HTBAC. Assuming
L.~\ref{lst:ties.py}, we run the runner retrieving runtime results, we
specify an adaptivity script for the evaluator, create TIES's \texttt{step1}.
The analysis script operates on partial simulation results, generating new
simulation conditions for the next simulation step.}]{ties_adaptivity.py}

% \jhanote{As written, this sub-section has nothing about HTBAC and would be
% better placed/merged in Science Driver.  What would be useful is
% pseudo-code or code listing to show how ESMACS and TIES are encoded in
% HTBAC}\jdnote{added context}

% While ESMACS and TIES protocols compute binding affinity calculations for
% different systems and purposes, their implementation in HTBAC have the same
% underlying pattern, consisting of simulations steps, followed by one or
% more analysis step(s). Designers of free energy protocols can utilize the
% simulation and analysis components of HTBAC to create any customized
% sequence of simulation(s) and/or analysis steps. The main implementation
% differences between protocols is in their design of ensemble members.

% Listing~\ref{lst:esmacs.py} shows the skeleton of the ESMACS protocol
% implemented in HTBAC. Specifically, the listing shows the customization of
% simulation conditions for a minimization step of the protocol.

% In \S\ref{ssec:adaptive_execution} we show HTBAC's adaptive execution
% capabilities. Here we show an example of a 5 step simulation run. For
% brevity, we did not specify the simulation conditions for
% \texttt{TIES.step0--2}, which are typically minimization and equilibration
% steps. We focus on implementing adaptivity within the protocol for the
% production MD step. In a non-adaptive protocol, the production MD step
% executes for the entire simulation duration as specified by the user. The
% adaptive protocol breaks down the production MD step into multiple, shorter
% steps. Between steps, the user assigns analysis scripts that generate
% simulation conditions necessary for further simulations. The shorter
% \texttt{TIES.step3} and \texttt{TIES.step4} specify production MD phase 1
% and phase 2, respectively. Simulations from \texttt{TIES.step3} execute but
% the application does not terminate, as specified by the flag
% \texttt{terminate = false}. The adaptive quadratures function computes
% where to place subsequent $\lambda$ windows and provides the information to
% \texttt{TIES.step4}.

% The additional windows



% ---------------------------------------------------------------------------
% S6
% ---------------------------------------------------------------------------
\section{Experiments}
\label{sec:experiments}
Computational campaigns for drug discovery require sizable compute resources
both in terms of scale and allocation. Typically, a computational campaign
explores a large number of drug candidates by running several workflows
multiple times, each requiring thousands of concurrent simulations. Before
embarking on a drug campaign that will utilize 150 million core hours on NCSA
Blue Waters, we perform experiments to characterize the weak and strong
scaling performance of HTBAC and its overheads on Blue Waters. We validate
the results of the free energy calculations produced using HTBAC by comparing
them to results that are already available in referenced literature.

Given that protocols like TIES are more computationally demanding than
protocols like ESMACS, it is paramount to use resource efficiently,
especially for campaigns that have a predefined computational budget. As
described in Sections~\ref{sec:science-drivers} and~\ref{sec:htbac}, adaptive
simulation methods have the potential for higher computational efficiency
when compared to non-adaptive simulation methods. Specifically, adaptive
simulation methods can reduce the number of simulations for each lead
candidate without a loss in free energy accuracy and with a lower
computational load. Further, given a defined number of resources, adaptive
methods can also increase computational efficiency by improving the rate of
statistical convergence and thereby reducing the time to solution.

Specifying, coordinating and executing adaptive simulation methods requires
middleware supporting runtime decisions. As seen in Section~\ref{sec:htbac},
HTBAC has been designed to enable decisions based on intermediate runtime
results. These results are evaluated to determine a redistribution of
resource to better support the execution of new simulations. We use HTBAC to
perform experiments about adaptivity, comparing and characterizing resource
efficiency with non-adaptive simulation methods.

% - Weak Scalability Design : Keep Pipeline of Ensembles to show barrier
%   needed in S5 and S6.
% - Performance, generality with weak scaling (agnostic to kernel) Added
%   functionality (do not speak about binding adaptivity to performance or
%   generality).
% - In use case: add in why TIES is challenging, and why adaptivity is
%   challenging.

% \mtnote{General note. \\
%         - Currently, this section is a merge of two distinct contributions.
%           These need to be better merged. For example, we should consider
%           whether (i) the description of the adaptive experiments should be
%           moved to the first subsection ``Experimental Setup''; (ii) Table 1
%           should be extended to include the adaptive/nonadaptive properties;
%           (iii) the discussion of the results should have a uniform
%           argumentative format; etc.\\
%         - The structure of the section should be revise making the division
%           in subsections consistent and changing/refining their naming.\\
%         - A.0,1,2 require major rewriting of the text and the writing of
%           relevant portions of new text. This includes: (i) eliminating
%           redundant (copied) text; (ii) Eliminating or referencing text
%           copied from other publications; (iii) reorganize the argumentation
%           following the other general note left in the comments and better
%           separating the experimental variables (especially including
%           adaptive/nonadaptive) properties; (iv) adding plots (even if with
%           incomplete results) and writing an actual analysis of the results
%           for each plot.}


% \jdnote{Revisions: \\ 
%         - Rewrote experimental setup by starting with motivation and
%           explaining overview of experiments
%         - Appended all experiments to table
%         - Created subsections for weak scaling, strong scaling, validation,
%           adaptive experiments
%         - Added plots (some are incomplete) and rearranged analysis
%         - Kristof added his data for adaptive experiments and
%           wrote results 
%         - Strong scaling subsection still waiting on data from ESMACS
%         - For now, removed anything related to Titan}

% \jdnote{should we mention anything about exascale?}\mtnote{I would not but
% I would quantify the scale at which we need to run. As it stands, this
% sentence appears too generic too me.} \jdnote{mentioned lower bound on
% concurrent number of simulations}

% \mtnote{what does this mean? We cited the results already in the paper?
% These results are published?}
% \mtnote{This sentence does not read well to me. I would refine, depending
% on the answer to the previous comment}.
% \jdnote{I meant to say we validate our free energy calculation results from
% this paper with known results . Does it make more sense now?}

% \mtnote{Grammar: more than?}\jdnote{more than ESMACS}
% \mtnote{Methods for? Maybe `simulation methods'?}

% \mtnote{simulation?}
% \mtnote{simulation?} 
 
% \mtnote{Note: experiments are not adaptive, they are about adaptive
% simulation methods. Analogously, experiments are not compared to
% non-adaptive experiments but they (should) offer comparable data about both
% types of methods.} \mtnote{Note: I rewrote the whole paragraph. The
% previous version was compressed and ``left too much in the authors'
% pens''.}

% ---------------------------------------------------------------------------
\subsection{Experiment Setup}\label{ssec:exp_design}

Table~\ref{tab:experiments} shows 7 experiments we designed to characterize
the behavior of HTBAC on Blue Waters. Each experiment executes the ESMACS
and/or TIES protocol for different physical systems. Experiments 1--5 use the
BRD4 physical systems provided by GlaxoSmithKline, while experiments 6 and 7
utilize the PTP1B, MC1, and TYK2 physical systems which are a diverse set of
proteins.

\begin{table*}
    \caption{Parameters of scalability and adaptivity experiments.}
    \label{tab:experiments}
    \centering
    \resizebox{\textwidth}{!}{
    \begin{tabular}{l            % Experiment ID
                    l            % experiment type
                    l            % physical system 
    %                l            % CI
                    l            % protocol type
                    l            % number of protocols
                    l            % total cores
                    }
    %
    \toprule
    %
    \B{ID}                            &  % Experiment ID
    \B{Type of Experiment}            &  % experiment type
    \B{Physical System(s)}            &  % physical system
    % \B{Computing Infrastructure (CI)} &  % CI
    \B{Protocol(s)}                   &  % protocol type
    \B{No. Protocol(s)}               &  % number of protocols
    \B{Total Cores}                   \\ % total cores
    %
    \midrule
    %
    \B{1}                             &  % Experiment ID
    Weak scaling                      &  % experiment type
    BRD4                              &  % physical system
    % Blue Waters                       &  % CI
    ESMACS                            &  % protocol type
    (2, 4, 8, 16)                     &  % number of protocols
    1600, 3200, 6400                  \\ % total cores
    % 
    \B{2}                             &  % Experiment ID
    Weak scaling                      &  % experiment type
    BRD4                              &  % physical system
    % Blue Waters                       &  % CI
    TIES                              &  % protocol type
    (2, 4, 8)                         &  % number of protocols
    4160, 8320, 16640                 \\ % total cores
    %
    \B{3}                             &  % Experiment ID
    Weak scaling                      &  % experiment type
    BRD4                              &  % physical system
    % Blue Waters                       &  % CI
    ESMACS + TIES                     &  % protocol type
    (2, 4, 8)                         &  % number of protocols
    5280, 10560, 21120                \\ % total cores
    %
    \B{4}                             &  % Experiment ID
    Strong scaling                    &  % experiment type
    BRD4                              &  % physical system
    % Blue Waters                       &  % CI
    TIES                              &  % protocol type
    (8, 8, 8)                         &  % number of protocols
    16640, 8320, 4160                 \\ % total cores
    %
    \B{5}                             &  % Experiment ID
    Strong scaling                    &  % experiment type
    BRD4                              &  % physical system
    % Blue Waters                       &. % CI
    ESMACS                            &  % protocol type
    (16, 16, 16)                      &  % number of protocols
    6400, 3200, 1600                  \\ % total cores
    %
    \B{6}                             &  % Experiment ID
    Non-adaptivity                    &  % experiment type
    PTP1B, MC1, TYK2                  &  % physical system
    % Blue Waters                       &  % CI
    TIES                              &  % protocol type
    (1, 1, 1)                         &  % number of protocols
    2080, 2080, 2080                  \\ % total cores
    %
    \B{7}                             &  % Experiment ID
    Adaptivity                        &  % experiment type
    PTP1B, MC1, TYK2                  &  % physical system
    % Blue Waters                       &  % CI
    TIES                              &  % protocol type
    (1, 1, 1)                         &  % number of protocols
    2080, 2080, 2080                  \\ % total cores
    %
    \bottomrule
    %
    \end{tabular}
    }
\up{}
\end{table*}

Experiment 1 and 2 measure the weak scalability of HTBAC using multiple
instances of the ESMACS (experiment 1) and TIES (experiment 2) protocol.
Experiments 3 uses both the TIES and ESMACS protocols, characterizing the
weak scaling of heterogeneous protocol executions. Experiments 4 and 5
measure the strong scalability of HTBAC using a fix number of instances of
the ESMACS (experiment 4) or TIES (experiment 5) protocol. Experiments 6 and
7 characterize nonadaptive and adaptive simulation methods using the TIES
protocol.

In each weak scaling experiment (1--3), we keep the ratio between resources
allocated and protocol instances constant. Consistently, for each experiment
we progressively increase both the number of cores (i.e., measure of
resource) and the number of protocol instances by a factor of 2. In each
strong scaling experiment (4--5), we change the ratio between resources
allocated and the number of protocol instances. For each experiment, we fix
the number of protocol instances and reduce the number of cores by a factor
of 2.

Weak scaling experiments provide insight into the size of the workload that
can be executed in a given amount of time, while strong scaling experiments
show how the time duration of the workload scales when adding resources. For
all the weak and strong scaling experiments we also characterize the
overheads of HTBAC, EnTK and RP. We also show an approximation of the time taken
by the resources to become available for each application. This offers insight 
about the impact of HTBAC and its middleware on the time to completion of each 
workload.

For weak and strong scaling experiments, we reduced the number of time-steps
of the protocols and omitted the analysis steps $S5$ and $S6$ of their
workflows (Figure~\ref{fig:ties_esmacs_application}). These simplifications
are consistent with characterizing scalability performance instead of
simulation duration. For the experiments 1--5 we used the following
time-steps: $S1=1000$; $S2=5000$; $S3=5000$; and $S4=50000$.

Experiments 6 and 7 compare the the accuracy and time to solution of
nonadaptive and adaptive simulation methods. For the nonadaptive simulation
method of Experiment 6 we use 13 preassigned and approximated $\lambda$
windows, consistent with the value reported in Ref.~\cite{Bhati2017}. In this
way, we produce 65 concurrent simulations for stages $S1$--$S4$ of the TIES
workflow (Figure~\ref{fig:ties_esmacs_application}). The production simulation 
stage $S4$ executes each simulation for 4ns. Stage $S5$ consists of 5 analysis 
tasks which aggregate the simulation results of $S4$. The global analysis stage 
$S6$ contains a single task that aggregates the results from $S5$.

In the adaptive workflow~\ref{fig:adaptive_ties}, we initialize the TIES
protocol with 3 $\lambda$ windows, obtaining 15 replicas. We separate stage
$S4$ of each TIES replica into 4 sub-stages. Each sub-stage runs a 1ns
simulation, followed by an adaptive quadratures analysis which estimates free
energy errors with respect to each interval of two $\lambda$ values. We
evaluate the total number of $\lambda$ windows as determined by the adaptive
quadrature results in order to measure the differences in time duration
between adaptive and non-adaptive methods as well as the differences in
accuracy.\mtnote{I had to simply this paragraph a lot. Please check whether I
omitted some essential detail or got the algorithm wrong.}
\jdnote{it works}

\begin{figure}
  \centering
  \includegraphics[width=\columnwidth]{figures/Adaptive_TIES.pdf}
  \caption{\mtnote{For quality reasons, it would be better to use 
  figures in pdf format.}\jdnote{addressed}. Illustrating the adaptive workflow. 
  After the 3 initial lambda windows are equilibrated, the first production stage 
  starts. This is followed by analysis at every lambda interval, to decide 
  whether to add a new window in the middle. The production-analysis is repeated 
  for 4 production steps in our implementation, not shown here.}
\label{fig:adaptive_TIES}
\end{figure}

For Experiment 6, we assigned the following simulation time-steps: $S1=3000$;
$S2=50000$; $S3=50000$; and $S4=2000000$. The adaptive simulation method of
Experiment 7 uses the same simulation time-steps, apart from $S4$ which is
divided into 4 sub-stages of 500000 time-steps each. 

% \jdnote{should we
% mention relationship between ns and time-steps?}

We performed all the experiments on Blue Waters, a 26868 node Cray XE6/XK6
SuperComputer with peak performance of 13.3 petaFLOPS managed by NCSA. We
initiated the experiments from a virtual machine outside NCSA. In this way,
we did not run any persistent process on the NCSA login node, consistent with
NCSA usage policies. We used HTBAC 0.1, EnTK 0.6, and RP 0.47 and the same
versions of the \texttt{NAMD-MPI} MD kernel, compiled according to the 
capabilities of each environment and of \texttt{NAMD} itself, and launched via 
the \texttt{aprun} command. For the analysis stages in the TIES protocol we used 
\texttt{AmberTools}.

NCSA sets a system policy on the maximum number of processes that
\texttt{aprun} can spawn, limiting the number of concurrent tasks we can
execute on Blue Waters to approximately 450. During the execution of
Experiment 2, we observed failing tasks once we scale up to 8 TIES protocol
instances, which is equivalent to 520 concurrent tasks. In a trial of 10
repetitions at this scale, we observed an average of 70 failing tasks, with a
standard deviation of 6.67, and range of 59 to 83 tasks. Given that we only
ran a small number of trials, more data would be required to properly model
the distribution type of these results.

NCSA allows to run only one MPI application for each compute node. As a
consequence, we had to run each MD simulation with 32 cores (i.e., one full
compute node) even if our performance measurement of \texttt{NAMD} on Blue Waters
indicated that 16 cores would offer the best trade-off between computing time
and communication overhead. 

% \jdnote{Should we include the scalability plot of
% a single NAMD task using aprun?}

% \mtnote{The following paragraphs are disconnected from the previous ones.
% We need a `joining' paragraph starting from something like: ``We designed n
% experiments\ldots We used m protocols\ldots See Table o\ldots''. Moved a
% paragraph here as a starting point.}\jdnote{provided ``segway"}

% Experiment 2 measures weak scaling performance of ESMACS at higher scales,
% using ORNL Titan.
% \mtnote{Would it be better to describe weak scaling in terms of solution
% time, number of resources and problem size, as done in the following
% paragraph?}\jdnote{moved to previous para when first introducing weak
% scaling}

% of the protocols \mtnote{which protocols? So far we wrote about methods
% (but we did not specify what type of methods are we referring to)}
% \jdnote{Does the modified description of the protocol types in the earlier
% paragraph give enough insight?} \mtnote{required by what/whom? ``keeping
% the number of pipelines fixed at the number required by \ldots'' or
% something like that?} \jdnote{I've changed this description of ws, do this
% sentence and the next suffice?}

% By design of each protocol, an increase in the number of instances means an
% increase in the number of pipelines. Further, \jdnote{mention the
% relationship between protocols and pipelines in section IV.}

% \mtnote{property of? I would replace with: ``weak scaling provides
% insight\ldots''} provides insight

% Next, we measure strong scaling performance of the ESMACS and TIES
% protocols in experiments 4 and 5, respectively, by fixing the number of
% protocol instances, but reducing the cores by a factor of 2 in each run.

% \mtnote{which ones?} by fixing the workload i.e. \mtnote{Grammar: please be
% careful about the use of `i.e.' as explained in my previous set of
% comments}
% \mtnote{Why do we need the `i.e.' at all? What does `workload' add?
% ``\ldots by fixing the number of pipelines of each protocol and reducing
% \ldots''}
% \jdnote{also changed this description of ss.}

% the number of cores by a factor of 2 in each run.

% \mtnote{what does `timing' mean in this context? Duration? }
% \jdnote{will be defined in section IV}\mtnote{we will define
% `what'?}\jdnote{ignore earlier comment, the correct phrasing is time
% duration of the workload, but I am not sure if this causes
% confusion}\jdnote{ also workload will be defined in Section IV}

% \jdnote{expecting linear speedup, and expect fixed overheads}\mtnote{is
% this comment about text that still needs to be written? If so, I would not
% mention expectations of results in the design section.}

% \mtnote{of?}\jdnote{addressed}

% Experiments 4 measures the strong scaling performance of ESMACS while
% experiment 5 repeats the same strong scaling experiment using the TIES
% protocol.

% \mtnote{This is a placeholder for an actual paragraph: note how the
% previous paragraphs have a structure based on numbering experiments (and
% referring them to the table via references that are still missing), while
% this and the following paragraphs do not use the same structure. Please
% expand as needed and note that a section is not `ready' until this kind of
% uniformity is not achieved.}\jdnote{I've added experimental design of
% adaptive/nonadaptive methods.}

% Keep iterating on the next two paragraphs to get the right description from
% Kristof

% We perform the experiments for the nonadaptive (experiment 6) and adaptive
% (experiment 7) simulation methods with the same number of cores and
% simulation duration (Table~\ref{tab:experiments}). By fixing these
% parameters, we enable the comparison between the accuracy and time to
% solution of the two methods.

% By design of the nonadaptive TIES workflow
% (Figure~\ref{fig:ties_esmacs_application}), each $\lambda$ window creates a
% new simulation condition that must be replicated. TIES requires 5 replicas
% per $\lambda$, thereby producing $\lambda \times 5$ or simulations.

% \mtnote{Reference table and exp number}
% \jdnote{addressed in this and previous sentence}
% \mtnote{Reference table with S*}
% \jdnote{the ref to Figure~\ref{fig:ties_esmacs_application} already points
% to the description of S*, no?}
% \mtnote{Grammar: no spaces before and after dashes}
% \jdnote{noted}
% \jdnote{reworded this paragraph}

% \mtnote{Grammar: it is not clear to what this `this' refers to} produces a
% total of 65 concurrent simulations for stages $S1$--$S4$ \mtnote{Reference
% table with S*} \mtnote{Grammar: no spaces before and after
% dashes}\jdnote{noted}. The production simulation stage $S4$ executes each
% simulation for 4ns. The analysis stage $S5$ assigns 5 analysis tasks which
% aggregate simulation results of $S4$ with respect to each replica. The
% global analysis stage $S6$ contains a single task that aggregates the
% results from $S5$.

% \mtnote{different from what?}\jdnote{changed}

% The estimation of the error determines whether
% \mtnote{Grammar: as already commented in the previous draft, `whether'
% implies already `or not'.\jdnote{noted} Adding it is
% redundant.}\jdnote{noted} to spawn additional $\lambda$ windows. If the
% error estimate in a certain interval is higher than the acceptable
% threshold, a new $\lambda$ window is assigned halfway between the interval.
% If the critical threshold is reached, we have reached convergence and the
% application stops; else, current simulations continue executing and new
% simulations begin executing based on newly generated $\lambda$ windows.

% The estimation computation \mtnote{This is too compressed. In itself,
% `estimation computation' means little. The reader has to go back to the
% previous paragraph to understand what those two words refer to. `We compute
% the free energy errors\ldots'?} is captured \mtnote{`captured' is
% metaphorical. Metaphors should be avoided when writing science. Also, the
% sentence is passive and also this should be avoided in scientific writing}
% as a single stage \mtnote{Does the reader know what a stage is?} containing
% a single task. By design of the adaptive quadrature algorithm \mtnote{where
% this algorithm is used? Does the reader know about it?}, the number of
% additional $\lambda$ windows will vary across candidates, which will impact
% the number of simulation-analysis iterations. \mtnote{And\ldots We are
% missing an explanation of why this is important from an experimental design
% point of view. Maybe because it changes the number of tasks?}
% \jdnote{will call this something else}.\jdnote{check with Kristof on the
% local and global analysis that happen for nonadaptive}

% \mtnote{Note: paragraphs like this indicate lack of iteration before
% releasing the manuscript for review. Too much is assumed, too little is
% converted from thoughts to written form. How the academic writing process
% works: Before releasing the manuscript for review, the author goes back and
% forth across the manuscript multiple times. Progressively, each paragraph
% is iterated until it is grammatically correct and it stands on its own
% content-wise. When a manuscript is received for review, the reviewer focus
% on the content of each paragraph and section, assessing their completeness
% and conceptual/formal correctness. Each grammatical error, each
% inconsistency or missing detail distracts the reviewer impeding her
% work.}\jdnote{I've removed the previous paragraph that gave trouble and
% updated the experimental setup of the adaptive and non-adaptive methods.}

% This procedure is repeated until convergence, at which point all concurrent
% simulation are terminated. We define convergence as the point in the
% production-analysis loop at which a desired error threshold is reached.

% When one protocol has converged whilst another has not, we can shift
% computational resources to favor simulations that require additional
% resources.

% individual\mtnote{I am not sure I understand what individual means in this
% context}

% strong scaling performance of ESMACS and TIES on Blue Waters. We fix the
% number of protocols and vary the amount of resources in order to produce
% generations

% \mtnote{The reader will struggle to understand what `generations' means in
% this context.} of execution.
% \mtnote{Two sizable paragraphs to describe weak scaling, four lines to
% describe strong scaling? I would expand.}

% \mtnote{This sentence seems to suggest that we run our experiments on a
% virtual machine. I know what you mean but the average reader has no idea
% about the capabilities of our middleware and the policies of not executing
% persistent processes on the head node of Blue Waters.}
% \jdnote{Can we omit this sentence?}

% and the Titan ones from an ORNL login node. as Blue Waters does not permit
% executing applications directly on the login node. The ORNL Titan
% experiments had instead to be performed from an ORNL login node.

% The minimization tasks of $S1$ to 1000 simulation time steps, while the
% equilibration tasks in $S2$ and $S3$ are \mtnote{Grammar: inconsistent
% tense. `Were' or `assign'} assigned 5000. The MD simulation tasks in $S4$
% are assigned 50,000 steps.

% \mtnote{when/why a simulation task is `production'? I am not sure I
% understand what that means}\jdnote{removed production for weak/strong.
% Production applies to adaptive/nonadaptive which need to show the
% production scale timesteps i.e. 2 or 3M}

% while on Titan a non-MPI, multi-core NAMD engine. On Titan we compiled NAMD
% with CUDA for the ESMACS experiments, and with OpenMP for the for TIES ones
% as NAMD does not support alchemical perturbations execution on GPU.

% application launch method, which is the NCSA Blue Waters designated command
% for describing the application parameters and resource requirements
% \mtnote{does the reader know what is a launch method?} \jdnote{better?}
% \mtnote{Unfortunately not: https://bluewaters.ncsa.illinois.edu/using-aprun
% . Alternative: `We perform (or performed depending on what tense you will
% decide to use) all Blue Waters experiments using the \texttt{aprun} command
% to launch the MD simulations'}.
% \jdnote{noted and addressed.} During our scalability experiments we found
% two main limitations.
% \mtnote{this is too generic. Based on this paragraph, we found two main
% limitations. This is what we should therefore write.}

% \jdnote{I've changed this again because the limitation on the max no. of
% concurrent CUs is a system setting (similar to what we experienced
% first-hand on Titan), but it's not a technical limitation of APRUN.}

% Unfortunately, \texttt{aprun} has several \mtnote{this is too generic.
% Based on this paragraph, we found two main limitations. This is what we
% should therefore write.} limitations: First, the number of concurrent tasks
% execution is limited to \ldots. On average, during our experiments we were
% not able to execute more than 436 concurrent tasks.
% \mtnote{how do we know this average?}.
% \jdnote{would it make sense to show a bell curve of the distribution of
% failing units? I have 8 data points, but I can run more and include Mark's
% Cray paper?}
% \mtnote{Agreed. The comment was about saying that you observed this average
% while performing experiments. See how I rewrote the paragraph}.
% \jdnote{added the data for the failing units however including a
% distribution of this data is not meaningful as the sample size in the
% number of trials is very small. By looking at some histograms where I vary
% the bin size, it could be either Gaussian or multimodal distribution. Only
% more data will tell us for sure what kind of distribution we have.
% Moreover, Mark's Cray paper shows range of concurrent units using ORTE-LIB,
% so I will not reference this here.}

% \mtnote{this is metaphorical. Please change using a factual predicate.}
% \jdnote{changed the phrasing a bit}
% \mtnote{Note: I had to rewrite the paragraph because the current version
% was too unrefined. Please see the note above about the need to
% progressively refine each paragraph.}

% From our own scalability performance measurements of NAMD on Blue Waters,
% we observe the ideal cores per task to be 16, however Blue Waters does not
% permit running multiple MPI applications on the same node, hence each NAMD
% task requires a complete node to maintain concurrency.
% \mtnote{This needs to go before previous sentence as it explains why we use
% 32 cores for each NAMD executable. Note: it is not true that NAMD
% `requires' 32 cores as stated above.}\jdnote{moved here since this para
% discusses \texttt{aprun} limitations}

% In order to continue execution on higher scales \mtnote{Why do we need an
% higher scale?}, we perform higher scales \mtnote{I am not sure what
% `performing higher scales' means} on ORNL Titan, using the ORTE launch
% method \mtnote{Why don't we use ORTE on BW? The reader will probably know
% that they are similar machines}.

% The difference in platforms produces overheads \mtnote{Only overheads or
% also different execution time of NAMD---seeing the differences described in
% the previous paragraph} that can be captured by RTS \mtnote{does the reader
% know what a RTS is?} and are shown \mtnote{careful with the use of
% `demonstrate', especially in a scientific paper} in the figures
% \mtnote{which figures?} as "APRUN overhead" and "ORTE overhead"
% \mtnote{LaTeX requires different quotes}.

% For tasks pertaining to $S1$ -- $S4$ \mtnote{What are $S1$ and $S4$? LaTeX
% wants double dash without space to indicate a range}, while the analysis
% stage, $S5$ use AmberTools \mtnote{grammar: I am not able to parse this
% sentence}.

% For both adaptive and nonadaptive experiments \mtnote{the reader knows
% about Experiment 1-7, not about adaptive/nonadaptive ones},

% In the nonadaptive experiments,

% For the adaptive experiments, each substage
% \mtnote{What is a `substage'?} of $S4$ i.e. $S4.1$--$S4.4$ is assigned
% 50,000 steps.\mtnote{Why the difference is number of steps and why it is
% relevant?}

% HTBAC submits a resource request to EnTK, to which EnTK uses RP to acquire
% resources via a single pilot. Accordingly, we request the maximum number of
% cores required by the workload as the number of cores in a pilot.

% To characterize the weak scaling performance of TIES \mtnote{I am not sure
% I understand this sentence: For measuring the weak scaling performance of
% TIES?}
% \jdnote{better?}, we use between 4160 and 33280 cores as indicated in
% Figure~\ref{fig:weak_scaling_TIES} because the NAMD executable used in all
% tasks \mtnote{does the reader know what a task is in this context?}
% \jdnote{will be defined in Section IV} from $S1$-$S4$ require
% \mtnote{grammar: requires} a single node i.e. 32 cores per task, as
% mentioned earlier in the APRUN limitation.
% \mtnote{why?}\jdnote{32 cores/task addressed by earlier mention of aprun
% MPI limit}

% For strong scaling performance, we fix the number of protocol instances to
% 8 instances given that we experience failing concurrency of tasks when we
% scale the workload to 8 protocols.\mtnote{why 8?} We vary the amount of
% resources as shown in Figure~\ref{fig:strong_scaling_TIES} by showing
% reducing of resources by a factor of 2. \mtnote{Please expand indicating
% and justifying the the chosen amount of resources. Experiments can be
% design in many ways, we need not only to describe but also to justify why
% we choose a specific design for our experiments.}\jdnote{better?}

% For the TIES protocol, each pipeline \mtnote{does the reader know what a
% pipeline is?}\jdnote{will be address in Section IV} consists of six stages
% \mtnote{simulation stages?}. Each of the simulation stages contains a task
% for every unique ($\lambda$, replica) combination \mtnote{does the reader
% understand this?}\jdnote{will be addressed better in section IV}. In the
% non-adaptive workflow \mtnote{we never used `workflow'. Previously we used
% non-adaptive (written as `nonadaptive') experiment. Note that we did not
% introduce adaptive/nonadaptive experiments} scenario, the first 11
% $\lambda$ windows
% \mtnote{does the reader know what a lambda window is?} consist of the
% following values: $L$ is a vector with
% \begin{flalign} L &= \{ x_i: x_i\in[0,1]\; and\; x_{i+1} = x_i + \delta \},
% where\ \delta\ is\ 0.1. %&$$L=\{ x_i: x_i\in[0,1]\; and\; x_{i+1} = x_i +
% \delta \}$$%, where $\delta$ is $0.1$.
% \end{flalign}

% We append two additional values on both ends of $L$, completing 13
% $\lambda$ windows. Each $\lambda$ window consists of five replicas.
% Therefore there are a total of 65 tasks for every simulation stage
% \mtnote{depending on the previous sections of the paper, this `therefore'
% may have to be better explained to the reader}. The production simulations
% stage, $S4$ as described in figure~\ref{fig:pst}\mtnote{missing figure}
% executes a 4 ns simulation duration \mtnote{grammar: executes a simulation
% for 4 ns? a simulation with a 4 ns duration?}. The analysis stages of the
% protocol reduce the number of tasks \mtnote{Why/how?}. The first analysis
% task consists of five tasks where each task performs an aggregate analysis
% over all $\lambda$ windows for each replica. The second analysis stage
% consists of one task that aggregates the previous results and computes a
% single average across all replicas.

%----------------------------------------------------------------------------

\subsection{Characterization of Scalability Experiments}

\mtnote{I would create a subsection with overheads, discussing their behavior
and nature while referencing both weak and strong scaling experiments.}

We perform 5 experiments to characterize the overheads of HTBAC, EnTK and the 
runtime system, and \texttt{aprun} as a function of weak and strong scaling 
properties. We compare these overheads to the total execution time of all the 
tasks to assess their overall impact on the application duration. 
The durations we measure are defined as:

\begin{itemize}
    \item Task Execution Time: Time taken by the task executables to run on the 
    CI. 
    \item HTBAC Overhead: Time taken to process the application, instantiate the 
    components and subcomponents, translate protocols into pipelines for EnTK, 
    and validate resource descriptions. 
    \item EnTK and RP Overhead: Time taken by ensemble management system and the 
    runtime system to submit and manage the execution of tasks. 
    \item \texttt{aprun} Overhead: Time taken by \texttt{aprun} to schedule
    tasks on a set of compute nodes. 
\end{itemize} 

Total Time to Execution \(TTX\) measures only the aggregated execution time of 
all tasks' executables. 
\mtnote{TTX should measure only the aggregated time spent executing the
tasks' executables. All the rest should be part of RP overhead} 
\jdnote{
This sentence doesn't need to go in but I wanted to clarify a point: I would 
like to know if we should use TTX as TTX or something else like Task Execution Time. 
We have TTX measured by events: \texttt{cu-exec-start} and \texttt{cu-exec-stop}
, since \texttt{aprun} doesn't allow further scrutiny using events like 
\texttt{app-start} until \texttt{app-stop}. 
We use the term TTX internally, but maybe for this paper we should call it task 
execution time. TTX sounds like time to execution which could interpret the 
entire application, no?}

Once the runtime system relinquishes control flow to \texttt{aprun}, the 
precise times of when \texttt{aprun} schedules the task on the compute
nodes and when the MD kernel begins execution cannot be measured. Instead we 
approximate the \texttt{aprun} overhead by measuring the difference between 
TTX/Task Execution Duration and the total \texttt{NAMD} execution time, provided
by the \texttt{NAMD} output logs. 

% We characterize the execution time of all three applications \mtnote{which one?},
% \jdnote{given that I mention what each experiment entails earlier, can I keep 
% the application general and say all 3?} and overheads of HTBAC, EnTK, RP and 
% \texttt{aprun}. 

% \jdnote{Since we are showing overhead component separate, can we remove the
% total overhead calculation?}

% The total time to completion (TTC) of each application \mtnote{of what?} 
% \jdnote{per application} \mtnote{you see the difference produced by my 
% editing?} can be expressed as $TTC = TTX + T_{O}$ where Total Time to Execution 
% \(TTX\) measures only the aggregated execution time of all tasks' executables. 
% \mtnote{TTX should measure only the aggregated time spent executing the
% tasks' executables. All the rest should be part of RP overhead}\jdnote{for
% discussion with Matteo 06/04}\mtnote{I am afraid I do not understand this
% comment} 




% % $T_{O}$ equates to the total overheads produced by HTBAC, EnTK, and
% % RP: $$T_{O} = T_{O}\textsuperscript{HTBAC} + T_{O}\textsuperscript{EnTK} +
% % T_{O}\textsuperscript{RP}$$ 




% \mtnote{As defined, this formula seems to add
% twice part of the RP overheads.}
% \jdnote{I think I'm missing something: The RP
% overhead is what I get from the pilot duration logs. Not sure where I'm
% duplicating}
% % \mtnote{If TTX is more than just the aggregated execution time of
% % all the tasks' executable, then TTX and $T_{O}$ both contain part of RP
% % overhead.}

\subsection{Weak Scaling and Performance Characterization}

% \mtnote{Is this meant to be a subsubsection?}\jdnote{changed weak, strong
% and adaptive experiments \& results to subsections}

We investigate the weak scaling of HTBAC using three
applications \mtnote{which ones?}\jdnote{addressed}: Experiment 
1 using TIES-only, Experiment 2 using ESMACS-only, and Experiment 3 using
both TIES and ESMACS configurations. Each application uses a different protocol 
configuration with a number of instances determined by the 
number of concurrent executions supported by \texttt{aprun}. In each experiment, 
we increase the number of instances of the protocol configuration, 
proportionally to the amount of requested resources. 

Experiment 1 in Figure~\ref{fig:ws} (a) measures the weak scalability of HTBAC 
with the TIES protocol. Each instance of the TIES protocol contains a single 
pipeline with 4 stages and 65 concurrent tasks. We increase the number of 
instances linearly, between 2 and 8. When scaling to 8 protocol instances, we 
execute beyond the average limit of 450 concurrent tasks. 
Nevertheless, we include results at this scale to show HTBAC's performance 
capabilities in supporting higher orders of concurrent simulations. 

Experiment 2 in Figure~\ref{fig:ws} (b) measures weak scalability with the 
ESMACS protocol. We increase the number of instances linearly, between 2 and 16. 
Each ESMACS protocol contains 1 pipeline with 4 stages and 25
concurrent tasks \mtnote{Do we `increase the number of instances linearly,
between 2 and 8' also in experiment 2? If so, we should say that only once
for Experiment 1--3. If not, why?}\jdnote{addressed the scaling ranges, and 
explained why}. 

Experiment 3 in Figure~\ref{fig:ws} (c) measures weak scalability
with instances of both TIES and ESMACS protocols. Also in this case, we scale
the instances of both protocols linearly, between 2 and 8. The first 
configuration shows 1 ESMACS and 1 TIES protocol, and with each increase in 
scale we double the number of protocols. Experiments 2 and 3 
show scalability ranges within the limit of the maximum number of concurrent 
tasks we can successfully execute on Blue Waters. For all weak
scaling experiments we use \mtnote{Grammar: conjugation}\jdnote{back to the 
present} physical systems from the \texttt{BRD4-GSK} library with the same 
number of atoms and similar chemical properties. The uniformity of these 
physical systems ensures a consistent workload that is insignificant to any 
variability in performance characterization. 
\jdnote{this is to say that the fluctuations in
execution times between data points are invariant to the physical
systems}\mtnote{Nice comment, I would edit it and put it in the text!}.
\jdnote{addressed}

% \mtnote{does the reader know what a task is in this context?} \jdnote{will
% be defined in Section IV}



% \mtnote{grammar: I do not understand the sentence after the `;'} given by
% the sum of the constituent overheads: $$T_{O} =
% T_{O}\textsuperscript{HTBAC} + T_{O}\textsuperscript{EnTK} +
% T_{O}\textsuperscript{RP}$$.


\dwwnote{The figure captions and legends don't appear to agree which axis 
applies to which measurement. Also the legend should be inside the plots.}
\jdnote{addressed, except with subplots I figured it would look cleaner with the
legend at the bottom.}

\begin{figure}
  \centering
    \includegraphics[width=\columnwidth]{figures/ws_all.pdf}
    \caption{Weak scaling properties of HTBAC. We
    investigate the weak scaling of HTBAC as the ratio of the number of
    protocol instances to resources is kept constant. Overheads of HTBAC, 
    EnTK + RP, and \texttt{aprun}, and Task Execution Time as a function of (a)
    TIES protocol (Experiment 1), (b) ESMACS protocol (Experiment 2) , and (c) 
    TIES and ESMACS protocols (Experiment 3). We ran 3 trials for each 
    experiment.} 
\label{fig:ws}
\end{figure}

\jdnote{Maybe combine captions into 1?}\mtnote{Agreed.}\jdnote{One figure, 3 
subplots, 1 caption is cleaner}

In all weak scaling experiments (Figure~\ref{fig:ws}) we
observe Task Execution Time to be fixed as the number of protocol increases, 
thereby showing ideal weak scaling behavior. \mtnote{Please refine the previous 
sentence as it might not say what I think you want to say: What does it mean 
that TTX is `within error margin'? And how this relates to `ideal weak scaling 
behavior'? Maybe the former applies to the latter instead of TTX?}.
\jdnote{does this read better?} The error bars are calculated
using 3 trials per protocol instance measure. While it is highly desirable to use 
larger trials to determine the precision of our observations, our current data
already suggests a trend of reproducibility. \mtnote{This might be consider
little without further explanation}\jdnote{how is this?}

The HTBAC overhead depends mostly on the number of protocol instances that
need to be generated for an application. This overhead shows a super linear
increase as we grow the number of protocol instances, but its duration is
negligible when compared to Task Execution Time.

% HTBAC enables concurrent execution of multiple protocol \mtnote{what type
% of protocols?} instances.

The runtime overheads, mainly EnTK and RP, mostly depend on the number of
tasks that need to be translated in-memory from a Python object to a
task description~\cite{dakka2017}. For RP this is confirmed in 
Ref.~\cite{merzky2018} 
\mtnote{This sentence is copied from
a previous paper we wrote: does the reader know what a CU is?}
\jdnote{I will add in a brief mention of CU in Section IV}\mtnote{I would
just use task as you do in the following sentence. IMHO, using CU adds yet
another pseudo-technical term as a synonym of a term we are already using,
i.e., task}\jdnote{noted and added reference for RP as well}.
As such, those overheads are expected to grow proportionally to
the number of tasks as observed in Figure~\ref{fig:ws}, bar in blue color, \ldots.
\mtnote{Please add reference to the relevant figure(s) and details.} 

\mtnote{I commented all the following. IMHO, it does not contribute to
understand the current analysis of the results. Do we need to expand the
analysis?}
% ------------------------- COMMENTED BY MT -------------------------
% EnTK submits CU descriptions to a database that is then pulled by
% RP.\mtnote{I am not sure this is accurate}\jdnote{took out "same" database}
% In addition, each stage constructed by EnTK maintains sequential barriers.
% \mtnote{Doe the reader know what a barrier is?}\jdnote{I will add this to
% the EnTK description in Section IV} RP remains dormant until completion of
% the current tasks before staging the next tasks. \mtnote{This figure is
% about scaling behavior, not about the mechanics of EnTK and RP here
% discussed.}\jdnote{took out figure ref}
% ------------------------- END COMMENTED BY MT ----------------------

% The impact of the synchronization barriers increases with the number of CUs
% as seen in the 16 protocol instances data point in
% This pull operation occurs over a wide area networks, which introduces
% varying amounts of latency.

The RP overhead is calculated measuring and aggregating the execution time
(i.e., duration) of the RP components that manage and coordinate the
execution of the application. Among these components, the task scheduler of
RP introduces the largest overhead. The task scheduling overhead scales
linearly with the number of tasks because the time taken to schedule a task
depends on the total number of tasks that need to be scheduled. 
% The main contributor to the increase in overhead in RP is based on the the
% time of resources inactivity while RP schedules new tasks. As such, the
% overhead is expected to grow proportionally to the number of concurrent
% tasks~\cite{dakka2017}.
\mtnote{I am afraid this whole paragraph needed to be
rewritten. I tried to fix it but we are missing a comparative evaluation of
this overhead. For example, does this overhead matter when compared to TTX or
TTC? Given a computation campaign with TIES and ESMACS, how much allocation
time would be spent on this overhead? What would be the total percentage of
allocation/resources spent in overheads?}

% Among these durations, the pilot duration contains the time to bootstrap
% and terminate the pilot. The task overheads contain the executor's spawning
% of the task, folder staging preparation, and operating system task
% spawning. \mtnote{We need to explain to the reader why all this is relevant
% in this context.} \jdnote{need to discuss on 06/04, should we keep these
% details or keep the focus on the paragraph just above this one?}

Furthermore, an additional overhead, driven by the \texttt{aprun} launch
method increases as we approach a system limit on the number of concurrent
\texttt{aprun} processes. For example, when scaling to 8 TIES protocol
instances as shown in Figure~\ref{fig:ws} (a), the application
requires 520 concurrent tasks. Considering that we observe \texttt{aprun}
failures at approximately  concurrent processes, this accounts for
additional time that is factored into Task Execution Time. 
\mtnote{We need to explain why TTX is affected. Are tasks restarted?
Also, does this overhead matter? See previous comment about RP overhead
impact on TTX/TTC/Resource utilization.}

% \mtnote{this sentence does not read well. I would refine it}
% \mtnote{this needs further explanation}
% \jdnote{reworded paragraph, also commented out next paragraph on EnTK
% synchronization until I can justify better.}

% \mtnote{Does the reader know what aprun is? Also, I would use something
% like textttt to identify the name of a command or
% executable}\jdnote{addressed}

% \mtnote{concurrent?}\jdnote{addressed} 

% Together, the EnTK-enabled synchronization barriers and \texttt{aprun}
% overhead failures \mtnote{until now, we have implicitly assumed that our
% overheads are measured in time. Did we now use also number of failures?}
% introduce delays in the scheduling of the CUs and results in higher
% overheads \mtnote{Is this sentence circular? Overhead increases
% contributing to increasing the overhead?}. Lastly, we notice that each
% additional protocol instance contributes to roughly 55 additional seconds
% in \(TTX\).\mtnote{All this needs to be clarified and expanded into an
% actual discussion of the results.}

% We use between 4160 and 16,640 cores as indicated in
% Figure~\ref{fig:weak_scaling_TIES}.

% To characterize the weak scaling performance of TIES \mtnote{I am not sure
% I understand this sentence: For measuring the weak scaling performance of
% TIES?}
% \jdnote{better?}, we use between 4160 and 33280 cores as indicated in
% Figure~\ref{fig:weak_scaling_TIES} because the NAMD executable used in all
% tasks \mtnote{does the reader know what a task is in this context?}
% \jdnote{will be defined in Section IV} from $S1$ -- $S4$ require
% \mtnote{grammar: requires} a single node i.e. 32 cores per task, as
% mentioned earlier in the APRUN limitation.
% \mtnote{why?}\jdnote{32 cores/task addressed by earlier mention of aprun
% MPI limit}

% With each new protocol instance generated for an application, the HTBAC
% overhead grows to match the additional requirement of generating new
% protocols.

% In order to understand the contribution of the various events in HTBAC,
% termed as HTBAC overhead, to

%----------------------------------------------------------------------------
% \subsubsection{Weak Scaling Experiments}

% We investigated the weak scalability properties for the TIES protocol by
% growing the number of protocol instances while adhering to the required
% number of pipelines. By design of each protocol, an increase in the number
% of instances simply means an increase in the number of pipelines. The first
% weak scalability experiment demonstrates the behavior of HTBAC, EnTK and RP
% using the multiple instances of the TIES protocol. By design of weak
% scaling, the ratio between the number of pipelines and cores are kept
% constant. As the number of cores (measure of resource) changes by a factor
% of 2, we introduce twice as many protocol instances. As designed, the weak
% scaling property provides insight into the size of the workload that can be
% investigated in a given amount of time.\mtnote{This paragraph is a copy of
% a previous paragraph. See comments and iterations of the previous paragraph
% before rewriting a new paragraph if needed.}

% ---------------------------------------------------------------------------
\subsection{Strong Scaling and Performance Characterization}

\mtnote{Should we have also a `Weak Scaling Experiments' header? Or maybe we
should change this one?}

To investigate strong scaling, we run two applications on Blue Waters: \ldots
and \ldots \mtnote{Please add name of the applications}. We investigate the
strong scaling behavior considering only homogeneous protocols. The
investigation of heterogeneous protocols is of interest but not at the risk
of cluttering the paper \mtnote{I am afraid we will need a better explanation
for this omission}. Both applications use the \texttt{BRD4-GSK} physical
systems \mtnote{Should this detail go in experiment design? This would apply
also to the weak scaling in the previous section}.

The first application \mtnote{name?} shows the strong scaling behavior of
TIES alone \mtnote{Are you sure this should not be the `nth experiment shows'
instead of the `first application shows'?}, using \texttt{NAMD-MPI} with 32
cores per task. We fix the number of instances of the TIES protocol to 8 due
to the described \texttt{aprun} limitations. We vary the amount of resources
between 4160, 8320 and 16640 cores. Given 4160 cores, we can execute 4
generations of 130 concurrent tasks; with 8320 cores, we can execute 2
generations of 260 concurrent tasks. The second application shows the strong
scaling behavior of the ESMACS protocol \mtnote{See previous comment about
application/experiment}. We fix the number of instances of the ESMACS
protocol to 16 and vary the amount of resources between 3200, 6400 and 12800
cores. This produces the same number of generations of execution \mtnote{add
reference to the subsection where generation is defined} as the first
application. \jdnote{will add the plot for ESMACS ss once I receive 2 full
trials}. As we increase the number of parallel resources,
Figure~\ref{fig:strong_scaling_TIES} shows linear speedup in \(TTX\) while
maintaining fixed $T_{O}$. The fixed $T_{O}$ suggests that the overheads are
mainly driven by the scheduling in the number of protocols and thereby the
number of tasks, not by the amount of resources. This is confirmed for RP in
Ref.~\cite{merzky2018}.\jdnote{I need to investigate this but I also presume
that aprun limits are also a factor of overhead.}

% The comparison between weak and strong scalability shows the overhead
% introduced by load balancing and scheduling tasks in multiple generations.

% By design of the TIES protocol, this workload show up to $65 \times 8$
% concurrent tasks, where each task uses 32 cores. The total number of
% concurrent

% \mtnote{See previous comments about terminology}\mtnote{I am not sure I
% understand: are we comparing overheads when weak and strong scaling our
% applications/workflows or are we first measuring the overheads when strong
% scaling and then comparing them to those of weak scaling?}. As we scale the
% number of generation of concurrent tasks executions, we half the resources
% allocated by the pilot \mtnote{this should be explained in the experiment
% setup and design}.

\mtnote{General note. Experiments about weak and strong scaling show the
behavior of our software stack with one or more workflows/workloads. The
question these experiments answer to is: Given a workload/workflow, does this
software stack scale weakly and/or strongly? This is why, in the analysis of
our data, we look at the linearity (or lack of thereof) of our plots. The
analysis of the overhead(s) answer to a different question: What part of a
measure---e.g., time to completion---depends on the properties of an element
used to produce that measure---e.g., a component of our software stack?
Usually, the analysis of the overheads is used to explain why we observe lack
of scalability (weak or strong) as represented by a superlinearity in our
plots. The discussion of our experimental results should clearly distinguish
these questions and properly relate the study of scalability to that of
overheads. At the moment, our discussion does not do all this.}\jdnote{I
reduced the discussion of scalability down to the properties that relate to
contributions of overheads and trends observed in TTX.}

\begin{figure}
  \centering
   \includegraphics[width=\columnwidth]{figures/new_ss_ties.pdf}
    \caption{Strong scaling properties of HTBAC using TIES protocol. We
    investigate the strong scaling of HTBAC with a fixed number of protocol
    instances while varying the amount resources. Overheads of HTBAC and
    runtime overhead (left) and \(TTX\) (right) for experimental
    configurations investigating the strong scaling of TIES. We ran 3 trials
    for each configuration.}
\label{fig:strong_scaling_TIES}
\end{figure}



%----------------------------------------------------------------------------
\subsection{Experiments Validation}

% HTBAC automates the process of calculating the binding affinity of
% protein-ligand complexes from reading the input to analyzing the final
% results.

In order to validate the correctness of the results produced using HTBAC and
the BRD4-GSK physical systems, we compare our results with those previously
cited \mtnote{This is unclear. Do you mean: results that have been previously
published? If so, where?}. These experiments \mtnote{What experiments? The
previous sentence writes about a comparison.} are vital to gain confidence in
the algorithm \mtnote{which algorithm?} and to prove that it is indeed
calculating the correct values.

The validation experiments were based on the original study of Wan et
al.~\cite{Wan2017brd4} \mtnote{Note: `et al.' is an abbreviation of 'et
alia', Latin for `and [et] others [alia]'. As such, the correct spelling is
`et al.'. It is now corrected both here and in the table caption}. We
selected a subset of the protein ligand systems that were the subject of that
study: ligand transformations 3 to 1, 4, and 7. We then performed a full
simulation on all 3 systems and calculated the binding affinity using HTBAC.

The results of our experiments, collected in Table~\ref{tab:exp2}, show that
all three $\Delta \Delta$G values are within error bars of the original study
using the TIES protocol, reinforcing \mtnote{validating? Based on what we
wrote before this would seem to contradict our intent} the fact that the
results produced using HTBAC are indeed correct
\mtnote{I am not sure `correct' here applies to the results. We may want to
have a look about the difference between validation and verification,
focusing on the separation between methodology and results.}.

\begin{table}
  \centering
  \caption{Comparison of the calculated binding free energies using HTBAC,
  from the original study by Wan et al. and experimental data. In
  principle,\mtnote{Do they use the same protocol? If there are differences
  we should explain them.} the two theoretical studies used the same
  protocol. This experiment proved that HTBAC implemented TIES correctly, as
  the calculated values are either the same or within error bar of the
  original study. All values are in \textbf{kcal mol\textsuperscript{-1}}.}  
  % \begin{tabular}{l@{\hskip 1in}r@{\hskip 0.2in}r@{\hskip 0.2in}r}
  \begin{tabular}{lrrr}
    \toprule
    System & HTBAC & Wan et. al & Experiment \\
    \midrule
    BRD4 \textbf{3 to 1} & \num{0.39 +- 0.10} &   \num{0.41 +- 0.04} &  \num{0.3 +- 0.09} \\
    BRD4 \textbf{3 to 4} & \num{0.02 +- 0.12} &   \num{0.01 +- 0.06} &  \num{0.0 +- 0.13} \\
    BRD4 \textbf{3 to 7} & \num{-0.88 +- 0.17} &  \num{-0.90 +- 0.08} & \num{-1.3 +- 0.11} \\
    \bottomrule
  \end{tabular}
  \label{tab:exp2}
\end{table}

%----------------------------------------------------------------------------
% \subsection{Adaptive Experiments}

% The TIES workflow can benefit from an adaptive execution environment to
% improve the efficiency and accuracy of result. In \emph{adaptive
% experiments} we implemented the adaptive quadrature algorithm specifically
% customized for biosimulations.

% In the adaptive workflow, we alter the number of $\lambda$ windows being
% simulated over the course of a protocol instance. The position of new
% $\lambda$ windows depends on the estimated error of the integral measured
% between adjacent windows. Increasing the number of $\lambda$ windows in
% regions of rapid change will increase the accuracy of the overall integral
% to a greater degree than an arbitrarily placed window. In order to
% adaptively add lambda windows, we need access to the $\partial
% U/\partial\lambda$ values during runtime. Therefore, we break down the
% single production simulation stage (S4) \mtnote{I would reference the
% figure with the workflow diagram here} from the nonadaptive workflow into
% multiple smaller stages \mtnote{how many?}, each running for 1 ns. Once
% each simulation is complete within a stage, a decision is made about
% whether more $\lambda$ windows are required, and, if so, where these
% windows should be placed.

% We start out the simulation with 5 replicas of 3 \mtnote{why emphasis for
% 3?}\jdnote{addressed} equally spaced $\lambda$ windows, and equilibrate
% them. Then we repeatedly execute shorter production simulations followed by
% an analysis phase which determines where to place new lambda windows. This
% procedure is repeated until convergence, at which point all concurrent
% simulation are terminated. We define convergence as the point in the
% production-analysis loop at which a desired error threshold is reached.

% The success of this algorithm is determined by the decision where
% additional $\lambda$ windows should be introduced. In adaptive quadrature,
% this decision is made by calculating an error estimate on the integral and
% comparing this estimate to a threshold value. Due to the stochastic nature
% of biosimulations, it is non-trivial to determine this error and, as a
% proof of concept, we simplified this decision to replicate pre-calculated
% results. In future studies, we plan to use a dynamic decision process.

% Inter-node communication introduces a constraint on the number of new
% $\lambda$ windows that can be added at each iteration. Simulations must run
% on an integer number of nodes to reduce the overhead of inter-node
% communication. This means that the number of new $\lambda$ windows (i.e.,
% the number of simulations) \emph{has} to be either doubled or left
% unchanged. If the number of windows is doubled, the number of nodes per
% simulation can be halved automatically. Our algorithm loops through the
% current $\lambda$ window pairs until this criterion is reached, forcefully
% adding more windows when needed.


% I don't think we need this equation here, it's too trivial.
% \begin{flalign} L &= \{ x_i: x_i\in[0,1]\; and\; x_{i+1} = x_i + \delta \},
% where\ \delta\ is\ 0.5. %&$$L=\{ x_i: x_i\in[0,1]\; and\; x_{i+1} = x_i +
% \delta \}$$%, where $\delta$ is $0.1$.
% \end{flalign}

% For every $\lambda$ window we initialize with five replicas therefore
% yielding a total of 15 tasks. We run 15 tasks for stages $S1$ through
% $S4.1$. Between stages $S4.1$ and $S4.3$ the number of $\lambda$ windows
% doubles for every stage, which doubles the total number of tasks. The last
% production simulation stage, $S4.4$, runs for the remaining 2 ns durations.

% Our experiments implement adaptive change in the $\lambda$ windows sampled
% and not the timing of execution. In this way, we introduce only a single
% \mtnote{additional?} degree of freedom relative to \mtnote{compared to?}
% our baseline ``non-adaptive" experiments. HTBAC provides the functional
% capability to adaptively determine the time at which the $\lambda$ windows
% are chosen. However, in this paper we do not investigate the impact of such
% adaptivity, as the objective is to determine the feasibility of adaptive
% execution and the scientific merit of adaptive decision making.

%----------------------------------------------------------------------------


% \begin{figure}
%   \centering
%     \includegraphics[width=\columnwidth]{figures/adaptive_vs_nonadaptive_ps
%     eudo.pdf}
%     \caption{Illustrating the adaptive vs. non-adaptive workflow given the
%     same number of core hours}
% \label{fig:adaptive_vs_nonadaptive_TIES}
% \end{figure}

% ---------------------------------------------------------------------------
\subsection{Adaptive Experiments}

\mtnote{Should this be Results as a whole? Or are we missing the results for all
the others?} \kfnote{I think results and experiments are one section. This
subsubsection is for the adaptive quadrature. There is a separate one for
adaptive termination.}\mtnote{I agree and both approaches would work for me. We
may want to be consistent across all experiments though: either we use a single
subsection illustrating the results of all the experiment or we discuss the
result of each experiment as we introduce it.} \kfnote{Okay. We are going
with the second approach. Describe experiment then results in the same
subsection.}

The aim of introducing adaptive quadrature to alchemical free energy
calculation protocols (e.g. TIES) is to reduce time to completion while
maintaining (or increasing) the accuracy of the results. Time to completion
is measured by the number of core hours consumed by the simulations. Accuracy
is defined as the error with respect to a reference value, calculated via a
dense $\lambda$ window spacing (65 windows). This reference value is used to
establish the accuracy of the non-adaptive protocol (which has 13 $\lambda$
windows) and the adaptive protocol (which has a variable number of $\lambda$
windows, determined at run time).

One of the input parameters of the adaptive quadrature algorithm is the
desired acceptable error threshold of the estimated integral. We set this
threshold to the error of the non-adaptive algorithm calculated via the
reference value. The algorithm then tries to minimize the number of $\lambda$
windows constrained by the accuracy requirement.

Table~\ref{tab:adapquad} shows the results of running adaptive quadrature on 5
protein ligand systems, comparing the \(TTX\) and accuracy versus the
non-adaptive case. The number of lambda windows are reduced on average by
\SI{32}{\percent}, hence reducing \(TTX\) by the same amount. The error on the
adaptive results is also decreased, on average by \SI{77}{\percent}. More
importantly, the error on all of the systems are reduced to below \num{0.2}
kcal/mol, which has recently been shown \cite{} to be the upper bound of
reproducibility across different MD engines. 

The TTX of the TYK2 L7-L8 system has increased for the adaptive run by 1
$\lambda$ window compared to the non-adaptive case. This is due to the
non-adaptive error begin very low, and matching that same accuracy required the
use of a large number of windows. Nonetheless due to the efficient placing of
the windows, the accuracy of the free energy still increased by
\SI{40}{\percent}.

\begin{table*}
  \caption{Comparing results of adaptive, non-adaptive and reference runs}
  \label{tab:adapquad}
  \begin{tabular}{lSSSSSS}
    \toprule
    {System}                               & 
    {Ref $\Delta \Delta$G}                 &
    {Non-adaptive $\Delta \Delta$G}        &
    {Adaptive $\Delta \Delta$G}            &
    {Num. of $\lambda$ windows}            &
    {Decrease in time to completion}       &
    {Increase in accuracy}                 \\
    \midrule
    {PTP1B L1-L2}   & 
    -58.51 & 
    -57.87(64) & 
    -58.60(9) & 
    10 & 
    \SI{23}{\percent} & 
    \SI{86}{\percent} \\
    %
    {PTP1B L10-L12} & 
    1.83   & 
    2.05(22) & 
    1.94(7)  & 
    6  & 
    \SI{54}{\percent} &
    \SI{68}{\percent} \\
    %
    {MCL1  L32-L38} & 
    2.13   & 
    2.33(20) & 
    2.14(1)      & 
    7  & 
    \SI{46}{\percent} & 
    \SI{95}{\percent} \\
    %
    {TYK2  L4-L9}   &
    -28.69 & 
    -28.25(44) & 
    -28.67(1)  & 
    7  & 
    \SI{46}{\percent} & 
    \SI{98}{\percent} \\
    %
    {TYK2  L7-L8}   & 
    4.97   & 
    4.92(5) & 
    5.00(3)      & 
    14 &  
    \SI{-8}{\percent} & 
    \SI{40}{\percent} \\
    \bottomrule 
    
  \end{tabular}
\end{table*}

Figure~\ref{fig:adapconv} \mtnote{missing figure? There is a table with the
label `adapquad' but not a figure.} \kfnote{Fixed} shows progression of the
algorithm, starting from 3 initial lambda windows, and at each iteration
bisecting the interval where the error is above the set threshold. For system X,
we can see as how the parts of the function that change less have relatively
fewer windows, while parts of greater change (usually, though not always around
the $\lambda=0.5$ mark) have more $\lambda$ windows to represent the shape of
the function more closely.

\kfnote{Move this to algorithm description} In previous studies \cite{} the
error estimation during the adaptive algorithm is done by calculating the
integral estimate using two methods, one lower and one higher in complexity,
then the difference between the two can constitute as a rough estimate of the
error on that interval.

\begin{figure}
  \begin{tikzpicture}
\begin{axis}[
  title style={align=center},
  title={Observed accuracy of adaptive vs. non-adaptive workflows\\for same simulation duration ({\SI{6}{\nano\second}}) },
  no markers,
  every axis plot/.append style={ultra thick},
  xlabel=Resource consumption (CPU-hours),
  ylabel=Error (kcal/mol),
  scaled ticks=false,
  yticklabel style={
  /pgf/number format/precision=3,
  /pgf/number format/fixed},
  scale=0.85,
  ]
  
  
\addplot+[color=Orchid, smooth] table [x=Resource consumption, y=Non-adaptive error, col sep=comma] {figures/non_adaptive_accuracy.csv};

\addplot+[color=YellowGreen, smooth] table [x=Resource consumption, y=Adaptive error, col sep=comma] {figures/adaptive_accuracy.csv};

\legend{Non-adaptive,Adaptive}

\draw[densely dotted, color=gray] (10586.058921984,0.049299384672436553) -- (12000,0.049299384672436553);
\draw[densely dotted, color=gray] (8143.1222476800012,0.0015858767437348931) -- (12000,0.0015858767437348931);

\draw[densely dotted, color=gray] (10586.058921984,0.08) -- (10586.058921984,-0.025);
\draw[densely dotted, color=gray] (8143.1222476800012,0.08) -- (8143.1222476800012,-0.025);
\draw[->] (10586.058921984,0.075) -- (8143.1222476800012,0.075) ;
\node[align = right] at (8700, 0.115) {Resource\\consumption\\decrease};

\node at (9200, 0.025) {\contour{white}{Error decrease}};
\draw[->] (11500,0.049299384672436553) -- (11500,0.0015858767437348931) ;


  
\end{axis}
\end{tikzpicture}

  \caption{\sout{This is a good plot!} Please provide a meaningful caption}
  \label{fig:adapconv}
\end{figure}

\begin{figure*}
  \begin{tikzpicture}
\begin{axis}[
  ybar,
  ymin=0,
  ylabel=Error (kcal/mol),
  x tick label style  = {text width=1.5cm,align=center},
  symbolic x coords={PTP1B L1-L2,PTP1B L10-L12,TYK2 L7-L8,TYK2 L4-L9,MCL1 L32-L38}
  ]
  
\addplot+[color=Orchid] table [x=System, y=Non-adaptive error, col sep=comma] {figures/savings.csv};

\addplot+[color=YellowGreen] table [x=System, y=Adaptive error, col sep=comma] {figures/savings.csv};


\legend{Non-adaptive,Adaptive}
  
\end{axis}
\end{tikzpicture}%
%
%
\begin{tikzpicture}
\begin{axis}[
  ybar,
  ymin=0,
  ylabel=Resource consumption (CPU-hours),
  x tick label style  = {text width=1.5cm,align=center},
  symbolic x coords={PTP1B L1-L2,PTP1B L10-L12,TYK2 L7-L8,TYK2 L4-L9,MCL1 L32-L38}
  ]
  
\addplot+[color=Orchid] table [x=System, y=Non-adaptive cpuh, col sep=comma] {figures/savings.csv};

\addplot+[color=YellowGreen] table [x=System, y=Adaptive cpuh, col sep=comma] {figures/savings.csv};


\legend{Non-adaptive,Adaptive}
  
\end{axis}
\end{tikzpicture}

\caption{\sout{This is a good plot!} Please provide a meaningful caption}
  \label{fig:savings}
\end{figure*}



% ---------------------------------------------------------------------------
% S7
% ---------------------------------------------------------------------------
\section{Discussion and Conclusion}
\label{sec:discussion}


Simulation protocols based on ensembles of multiple runs of the same system
provide an efficient method for producing robust free energy estimate, and
equally important statistical uncertainties. Variations in the chemical and
biophysical properties of different systems impact the optimal protocol choice
for different proteins and classes of drugs targetting them. However, the optimal protocol for a given system 
is difficult to determine {\it a priori}, thus requiring runtime adaptation
to workflow executions. We introduce the High-throughput Binding Affinity
Calculator (HTBAC) to enable the scalable, adaptive and automated calculation
of the binding free energy on high-performance computing resources.

In this paper we demonstrate: (i) How HTBAC allows the concurrent screening
for drug binding affinities of multiple compounds at unprecedented scales,
both in the number of candidates and resources utilized. Specifically, we
investigated weak scaling behaviour for screening sixteen drug candidates
concurrently using thousands of multi-stage pipelines on more than 32,000
cores. This permits a rapid time-to-solution that is essentially invariant
with respect to the calculation protocol, size of target system and number
of ensemble simulations. (ii) The validation of binding free energies computed
using HTBAC with both experimental and previously published computational
results; (iii) HTBAC enabled the adaptive execution of the TIES protocol
providing greater convergence (i.e., lower errors) for a given amount of
computational resources. To the best of our knowledge, adaptive TI protocols
have not been benchmarked against non-adaptive implementations.

%nor have they been implemented at such scales before.



%HTBAC can also support a wide range of adaptivite scenarios.

As such, HTBAC advances the state of the scale and sophistication of binding
affinity calculation. In addition to reporting increasingly sophisticated
adaptive scenarios, in future, we will extend HTBAC to support the ``design of
experiments", facilitating optimization at the level of the overall
computational campaign and time-to-insight for a large database of drug
candidates, as opposed to for single simple calculations.


% HTBAC uses readily available
% building blocks to attain both workflow flexibility and performance; our
% scaling experiments are performed on the Blue Waters machine at NCSA.

\newpage

% ---------------------------------------------------------------------------
% BIBLIOGRAPHY
% ---------------------------------------------------------------------------
\bibliographystyle{IEEEtran}
\bibliography{rutgers,ucl}

\end{document}
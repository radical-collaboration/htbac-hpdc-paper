

Simulation protocols based on ensembles of multiple runs of the same system
provide an efficient method for producing robust free energy estimate, and
equally important statistical uncertainties. Variations in the chemical and
biophysical properties of different systems impact the optimal protocol choice
for different proteins and classes of drugs targetting them. However, the optimal protocol for a given system 
is difficult to determine {\it a priori}, thus requiring runtime adaptation
to workflow executions. We introduce the High-throughput Binding Affinity
Calculator (HTBAC) to enable the scalable, adaptive and automated calculation
of the binding free energy on high-performance computing resources.

In this paper we demonstrate: (i) How HTBAC allows the concurrent screening
for drug binding affinities of multiple compounds at unprecedented scales,
both in the number of candidates and resources utilized. Specifically, we
investigated weak scaling behaviour for screening sixteen drug candidates
concurrently using thousands of multi-stage pipelines on more than 32,000
cores. This permits a rapid time-to-solution that is essentially invariant
with respect to the calculation protocol, size of target system and number
of ensemble simulations. (ii) The validation of binding free energies computed
using HTBAC with both experimental and previously published computational
results; (iii) HTBAC enabled the adaptive execution of the TIES protocol
providing greater convergence (i.e., lower errors) for a given amount of
computational resources. To the best of our knowledge, adaptive TI protocols
have not been benchmarked against non-adaptive implementations.

%nor have they been implemented at such scales before.



%HTBAC can also support a wide range of adaptivite scenarios.

As such, HTBAC advances the state of the scale and sophistication of binding
affinity calculation. In addition to reporting increasingly sophisticated
adaptive scenarios, in future, we will extend HTBAC to support the ``design of
experiments", facilitating optimization at the level of the overall
computational campaign and time-to-insight for a large database of drug
candidates, as opposed to for single simple calculations.


% HTBAC uses readily available
% building blocks to attain both workflow flexibility and performance; our
% scaling experiments are performed on the Blue Waters machine at NCSA.
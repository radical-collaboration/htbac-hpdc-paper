Recent developments in both algorithms and architectures (in particular
GPGPUs) have led to increasing interest in the use of molecular simulation to
estimate the strength of macromolecular binding free
energies~\cite{DeVivo2016}. However, for molecular simulations to truly
influence decison making in industrial and clinical settings, the dual
challenges of scale (concurrent screening of drug candidate using thousands of
concurrent multi-stage pipelines) and sophistication (adaptive sampling or
selection of binding affinity protocols based upon system behaviour and
statistical uncertainty) will need to be tackled.

% which provides a domain specific interface to leverage recent advances in
% abstracting the relatonships between building blocks in workflow systems to

Tools that facilitate the automated scalable and sophisticated computation of
varied binding free energy calculations on high-performance computing
resources are necessary. We have previously developed a tool for automating
system building and simulation input generation, the Binding Affinity
Calculator (BAC)~\cite{Sadiq2008}. More recently we have introduced the High-Throughput Binding Affinity Calculator (HTBAC)~\cite{dakka2017}, which brings
advances in the design of domain-specific workflow systems using building
blocks to facilitate the automation of deployment and execution of rapid and
accurate calculation of binding affinities on leadership class machines. In
Dakka \textit{et al.}~\cite{dakka2017} we demonstrated how HTBAC scales almost
perfectly to hundreds of concurrent multi-stage pipelines for a single
protocol.
%, which permits rapid time-to-solution that is essentially invariant
%of the size of candidate ligands.

% as well as the type and number of protocols concurrently employed.

%In this paper we aim to reproduce a collaboration project between UCL and
%GlaxoSmithKline to study a congeneric series of drug candidates binding to the
%BRD4 protein 
%(inhibitors of which have shown promising preclinical efficacy in
%pathologies ranging from cancer to inflammation)
%~\cite{Wan2017brd4}. This
%study compared two different protocols, known as TIES and
%ESMACS~\cite{Wan2017brd4, Bhati2017}, both based on an ensemble simulation
%philosophy. 
In this paper we reproduce results from a study of  congoneric series of componds which employed 
two different protocols, known as TIES and
ESMACS~\cite{Wan2017brd4, Bhati2017}, both based on 
%an ensemble simulation
%philosophy.
multiple simulations of the same system.
%In this approach multiple simulations are executed based on the
%same input system description, providing enhanced sampling and reduced time to
%completion. 
TIES is rigorous but computationally expensive and has a limited range of 
applicability.
ESMACS is approximate but can be employed across any set of ligands at lower computational cost.
%TIES is based on rigorous, but computationally expensive,
%calculations of relative free energies (i.e. results provide a comparison
%between two drugs). ESMACS, in contrast, provides absolute binding free
%energies at low computational cost, but to achieve this coarse grains many of
%the details of the system being studied. The simplifications employed by
%ESMACS can reduce its effectiveness in some systems. 
%In the real world
%application of these technologies, d
Drug design projects have limited resources
and must make trade-offs between the needs for rigour and coverage of a wide
range of chemical space. Initially large numbers of compounds must be screened
to eliminate poor binders (using ESMACS), later more accurate methods (such as
TIES) are needed as good binders are refined and improved. %This means that
%many projects will combine the use of both protocols.

%We explore the use of HTBAC for aforementioned sophisticated exploration of
%novel compounds binding to a target protein. 
In order to support such
investigations, in addition to being scalable, HTBAC must be enhanced to
support flexible resource reallocations schemes where resources can be moved
between simulations run using different protocols or systems, for example,
when one calculation has converged whilst another has not. This adaptability
makes it easier to manage complex programmes where efficient use of resources
is required in order to achieve a time to completion of studies comparable to
those of high throughput chemistry. 
\dwwnote{Could the next sentence go?}
This functionality provides the
foundations upon which we develop adaptable simulation schemes that
automatically handle different system characteristics.

In this work we demonstrate the use of HTBAC to adaptively run both protocols,
including mixed protocol runs. 
Calculations using either protocol require simulations of free and 
protein bound ligand necessitating the coordination of runs with 
heterogeneous computational requirements.
%Executing either protocol often involves
%running simulations of both free and  bound  ligand, with the later involving
%a much larger system than the former. This means that even single calculations
%must coordinate simulations with heterogeneous computational requirements.
Further to the use of a common framework to improve the ease of deployment and
efficiency of execution of existing protocols we show how HTBAC can aid the
development of enhanced approaches. 
We demonstrate this capacity through the development of a variant of TIES 
employing adaptive sampling which improves time to convergence.

%The TIES protocol is highly sensitive to
%the chemical details of the compounds being studied, which means that
%different runs may require different sampling strategies to achieve optimal
%time to convergence. We have developed an adaptive variant of TIES which
%automatically increases the sampling in areas where it is needed for each
%system, allowing more rapid convergence of calculations. Furthermore, this
%approach facilitates a data driven approach to future protocol refinement.

%This approach not only allows us to tailor the protocol to the particular
%system but to provide meaningful statistical uncertainties for all of our
%models. 
These developments fit into a wider vision in which the use of
flexible and responsive computational protocols produce accurate,
precise and reproducible estimates of the free energy of binding with meaningful error bars. Not only
would this allow for wider uptake of computational techniques in industrial
settings but opens up possibilities of using these technologies in clinical
decison support scenarios. By creating a `digital twin', where the target
protein is based on the real patients genetic sequence, a specific individuals
response to different treatements could be predicted. 
\dwwnote{Could remove the next sentence}
This approach would be
applicable even in the case of rare variants where insufficient data is
available for statistical methods to be informative.

The novel contributions of HTBAC are: (i) Unprecedented throughput: it allows
the concurrent screening for drug binding affinities of multiple compounds at
unprecedented scales both in the number of candidates and resources utilized;
(ii) Agile selection of different binding affinity protocols: Inter-protocol
adaptivity: selection of ``optimal" protocol (as determined by accuracy and
computaional intensity); (iii) Intra-protocol Adaptivity: efficient execution
of individual protocols; and, (iv) Sets the stage for design of experiments
that will allow for optimization of the collective and overall  
time-to-insight as opposed to any single simple calculation.

\dwwnote{Do we need the next paragraph? A slightly shortened version is: 
In this paper we descibe the background to our binding free energy protocols and how we designed HTBAC to 
meet the challenges faced in bringing this approach up to extreme scale. We then describe a set of experiments 
characterizing the performance and
scalability of HTBAC and the development of an adaptive version of TIES.
}

In the next section, we review previous work using ensemble molecular dynamics
and outline the challenges faced in bringing this approach up to extreme
scale. Section 3 provides details of the TIES and ESMACS protocols and the
BRD4 system we will apply them to. In Section 4, we discuss how HTBAC has
been designed and implemented in order to meet the computational challenges
associated with the scalable execution of multiple, and possibly concurrently
executing protocols. In Sections 5 and 6 we describe the design and then
results of a series of experiments characterizing the performance and
scalability of HTBAC on the Blue Waters and Titan supercomputers. We conclude
with a discussion of the impact of HTBAC, implications for binding affinity
calculations and near-term future work.

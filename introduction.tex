In recent years, developments in both algorithms and hardware (in particular 
GPGPUs) have led to increasing interest in the use of molecular simulation to 
estimate the strength of macromolecular binding free energies \cite{DeVivo2016}. 
The same molecular simulation technologies that can be employed both in the
design new therapeutics and to investigate the origins of drug resistance in 
pathologies such as cancer and HIV. 

Creating simulation protocols which have well defined uncertainty and produce
statistically meaningful results represents a significant computational challenge. 
Furthermore, it is highly likely that differences among investigated systems will 
demand different protocols and that changes may be requires as studies progress. 
For example, drug design programmes often require the rapid screening of thousands 
of candidate compounds to filter out the worst binders before using more sensitive 
methods to refine the structure. 
Not all changes induced in protein shape or behavior are local to the drug binding 
site and, in some cases, simulation protocols will need to adjust to account for 
complex interactions between drugs and their targets within individual studies.

For molecular simulations to truly influence decison making in industrial and 
clinical settings, the dual challenges of
scale (thousands of concurrent multi-stage pipelines) and sophistication
(adaptive selection of binding affinity protocols based upon system behaviour and statistical uncertainty) will need to be tackled. 
Tools that facilitate the scalable and automated computation of varied binding 
free energy calculations on high-performance computing resources are necessary. 
To achieve that goal, we introduce the High-Throughput Binding Affinity Calculator 
(HTBAC).
HTBAC provides a domain specific interface to leverage recent advances in abstracting 
the relatonships between building blocks in workflow systems to facilitate the building 
and automation of rapid and accurate calculation of binding affinities.
We demonstrate how HTBAC scales almost perfectly to hundreds of concurrent
pipelines of binding affinity calculations on a leadership class machine. 
This permits the rapid time-to-solution that is essentially invariant of the size
of candidate ligands as well as the type and number of protocols concurrently
employed.

In this work we explore the use of HTBAC in a prototypical exploration of novel 
compounds binding to a target protein.
In this case we look to reproduce a collaboration run between UCL and GlaxoSmithKline to study a congeneric series of drug candidates binding to the BRD4 protein (inhibitors of which have shown promising preclinical efficacy in pathologies ranging from cancer to
inflammation)~\cite{Wan2017brd4}.


%Recent work that used molecular simulations to provide input to machine
%learning models~\cite{Ash2017} required simulations of 87 compounds even if
%they were designed merely to distinguish the highly active from weak
%inhibitors of the ERK2 kinase. If we wish to build on such studies to help
%inform later stages of the drug discovery pipeline, in which much more subtle
%alterations are involved, it is likely a much larger number of simulations
%will be required. This is before we begin to consider the influence of
%mutations or the selectivity of drugs to the more than 500 different
%genes in the human kinome~\cite{Li2016}.

In the next Section, we provide details of ensemble molecular dynamics
approach and its advantages over the single trajectory
approach. 
We also introduce the ESMACS and related protocols to compute
binding affinities using ensemble based approaches. 
In Section 3, we discuss
the computational challenges associated with the scalable execution of
multiple, and possibly concurrently executing protocols. 
Section 4 introduces
RADICAL-Cybertools -- a suite of building blocks to address the challenges
outlined in Section 3. 
In Section 5 we introduce HTBAC and describe how it
uses RADICAL-Cybertools to manage the execution of binding affinity calculations
at extreme scales. 
Experiments to characterize the performance and scalability
of HTBAC on the Blue Waters supercomputer are discussed in Section 6. 
We
conclude with a discussion of the impact of HTBAC, implication for binding
affinity calculations and near-term future work.

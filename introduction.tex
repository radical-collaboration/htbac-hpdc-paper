In recent years, developments in both algorithms and hardware (in particular
GPGPUs) have led to increasing interest in the use of molecular simulation to
estimate the strength of macromolecular binding free energies
\cite{DeVivo2016}. The same molecular simulation technologies can be employed
both in the design of new therapeutics and the investigation of the origins of
drug resistance in pathologies such as cancer and HIV.

For molecular simulations to truly influence decison making in industrial and
clinical settings, the dual challenges of scale (thousands of concurrent
multi-stage pipelines) and sophistication (adaptive sampling or selection of
binding affinity protocols based upon system behaviour and statistical
uncertainty) will need to be tackled. 

\jhanote{Proposed new opening in next paragraph. OK?}

Recent developments in both algorithms and architectures (in particular
GPGPUs) have led to increasing interest in the use of molecular simulation to
estimate the strength of macromolecular binding free energies
\cite{DeVivo2016}. However, for molecular simulations to truly influence
decison making in industrial and clinical settings, the dual challenges of
scale (thousands of concurrent multi-stage pipelines) and sophistication
(adaptive sampling or selection of binding affinity protocols based upon
system behaviour and statistical uncertainty) will need to be
tackled~\cite{XX}.

% which provides a domain specific interface to leverage recent advances in
% abstracting the relatonships between building blocks in workflow systems to

Tools that facilitate the scalable, automated and sophisticated computation of
varied binding free energy calculations on high- performance computing
resources are necessary. Towards that goal, we recently introduced the High-
Throughput Binding Affinity Calculator (HTBAC)~\cite{XX}, which brings
advances in the design of domain-specific workflow systems using building
blocks to facilitate the building and automation of rapid and accurate
calculation of binding affinities. Ref.~\cite{XX} we demonstrated how HTBAC
scales almost perfectly to hundreds of concurrent binding affinity
calculations on a leadership class machine. This permits the rapid time-to-
solution that is essentially invariant of the size of candidate ligands as
well as the type and number of protocols concurrently employed.

In this case we look to reproduce a collaboration run between UCL and
GlaxoSmithKline to study a congeneric series of drug candidates binding to the
BRD4 protein (inhibitors of which have shown promising preclinical efficacy in
pathologies ranging from cancer to inflammation)~\cite{Wan2017brd4}. This
study compared two different protocols, known as TIES and ESMACS, both based
on an ensemble simulation philosophy. In this approach multiple simulations
are executed based on the same input system description, providing enhanced
sampling and reduced time to completion. TIES is based on rigourous, but
computationally expensive, calculations of relative free energies (i.e.
results provide a comparison between two drugs). ESMACS, in contrast, provides
absolute binding free energies at low computational cost, but to achieve this
coarse grains many of the details of the system being studied. The results of
the simplifications employed by ESMACS is that its performance is heavily
system dependent. In the real world application of these technologies, drug
design projects have limited resources and must make trade-offs between the
needs for rigour and coverage of a wide range of chemical space. Initially
large numbers of compounds must be screened to eliminate poor binders (using
ESMACS), later more accurate methods (such as TIES) are needed as good binders
are refined and improved. This means that many projects will combine the use
of both protocols.

We explore the use of HTBAC for aforementioned sophisticated
exploration of novel compounds binding to a target protein.
In order to support such investigations HTBAC must be enhanced to support
% In this paper, we demonstrate how HTBAC facilitate 
flexible resource reallocations schemes where resources can be moved between
simulations run using different protocols or systems, for example, when one
calculation has converged whilst another has not. This adaptability makes it
easier to manage complex programmes where efficient use of resources is
required in order to achieve a time to completion of studies comparable to
those of high throughput chemistry. This functionality provides the
foundations from which we develop adaptable simulation schemes which
automatically handle different system characteristics.

In this work we demonstrate the use of HTBAC to adaptively run both protocols,
including mixed protocol runs. Both protocols can involve the running sets of
simulations which require very different levels of computational power.
Further to the use of a common framework to improve the ease of deployment and
efficiency of execution of existing protocols we show how HTBAC can aid the
development of enhanced approaches. The TIES protocol is highly sensitive to
the chemical details of the compounds being studied, which means that
different runs may require different sampling strategies to achieve optimal
time to convergence. We have developed an adaptive variant of TIES which
automatically increases the sampling in areas where it is needed for each
system, allowing more rapid convergence of calculations. Furthermore, this
approach facilitates a data driven approach to future protocol refinement.

\jhanote{Mention how this work brings us a step closer to realizing the
concept of "digital twin" i.e., computational modeling of experiments with
high-fidelity}

In the next section, we review previous work using ensemble molecular dynamics
and outline the challenges faced in bringing this approach up to extreme
scale. In Section 3, we discuss how HTBAC has been designed and implemented in
order to meet the computational challenges associated with the scalable
execution of multiple, and possibly concurrently executing protocols. Section
4 provides details of the TIES and ESMACS protocols and how these have been
mapped to teh abstractions underlying HTBAC. In Sections 5 and 6 we describe
the design and then results of a series of experiments characterizing the
performance and scalability of HTBAC on the Blue Waters and Titan
supercomputers. We conclude with a discussion of the impact of HTBAC,
implications for binding affinity calculations and near-term future work.

\documentclass[conference]{IEEEtran}
\usepackage{amsmath}
\usepackage{amssymb}
\usepackage{array}
\usepackage{booktabs}
\usepackage{color}
\usepackage{float}
\usepackage{graphicx}
\usepackage{ifpdf}
\usepackage[utf8]{inputenc}
\usepackage{keyval}
\usepackage{listings}
\usepackage{moresize}
\usepackage{multirow}
\usepackage[numbers,sort&compress,square]{natbib}
\usepackage{paralist}
\usepackage{rotating}
\usepackage{soul}
% \usepackage[style=base]{subcaption}
\usepackage{srcltx}
\usepackage{url}
\usepackage[dvipsnames]{xcolor}
\usepackage{xspace}
\usepackage{wrapfig}
\usepackage{hyperref}
% \usepackage{caption}
\usepackage{enumitem}

\definecolor{listinggray}{gray}{0.95}
\definecolor{darkgray}{gray}{0.7}
\definecolor{commentgreen}{rgb}{0, 0.4, 0}
\definecolor{darkblue}{rgb}{0, 0, 0.6}
\definecolor{purple}{rgb}{0.6, 0, 0.6}
\definecolor{middleblue}{rgb}{0, 0, 0.75}
\definecolor{darkred}{rgb}{0.4, 0, 0}
\definecolor{brown}{rgb}{0.5, 0.5, 0}
\definecolor{dkgreen}{rgb}{0,0.5,0}
\definecolor{orange}{rgb}{1,.5,0}
\definecolor{dandelion}{cmyk}{0,0.29,0.84,0}

\lstset{ 
  backgroundcolor=\color{white},   % choose the background color; you must add \usepackage{color} or \usepackage{xcolor}; should come as last argument
  basicstyle=\ttfamily\footnotesize,        % the size of the fonts that are used for the code
  breakatwhitespace=false,         % sets if automatic breaks should only happen at whitespace
  breaklines=true,                 % sets automatic line breaking
  captionpos=b,                    % sets the caption-position to bottom
  commentstyle=\color{purple},    % comment style
  deletekeywords={...},            % if you want to delete keywords from the given language
  escapeinside={\%*}{*)},          % if you want to add LaTeX within your code
  extendedchars=true,              % lets you use non-ASCII characters; for 8-bits encodings only, does not work with UTF-8
  % frame=single,                    % adds a frame around the code
  keepspaces=true,                 % keeps spaces in text, useful for keeping indentation of code (possibly needs columns=flexible)
  keywordstyle=\color{orange},       % keyword style
  language=python,                 % the language of the code
  morekeywords={*,...},            % if you want to add more keywords to the set
  numbers=left,                    % where to put the line-numbers; possible values are (none, left, right)
  numbersep=5pt,                   % how far the line-numbers are from the code
  numberstyle=\tiny\color{darkgray}, % the style that is used for the line-numbers
  rulecolor=\color{black},         % if not set, the frame-color may be changed on line-breaks within not-black text (e.g. comments (green here))
  showspaces=false,                % show spaces everywhere adding particular underscores; it overrides 'showstringspaces'
  showstringspaces=false,          % underline spaces within strings only
  showtabs=false,                  % show tabs within strings adding particular underscores
  stepnumber=2,                    % the step between two line-numbers. If it's 1, each line will be numbered
  stringstyle=\color{commentgreen},     % string literal style
  tabsize=2,                       % sets default tabsize to 2 spaces
  % title=\lstname                   % show the filename of files included with \lstinputlisting; also try caption instead of title
}

\usepackage[normalem]{ulem}
\makeatletter
\def\cyanuwave{\bgroup \markoverwith{\lower3.5\p@\hbox{\sixly \textcolor{cyan}{\char58}}}\ULon}
\def\reduwave{\bgroup \markoverwith{\lower3.5\p@\hbox{\sixly \textcolor{red}{\char58}}}\ULon}
\def\blueuwave{\bgroup \markoverwith{\lower3.5\p@\hbox{\sixly \textcolor{blue}{\char58}}}\ULon}
\font\sixly=lasy6 % does not re-load if already loaded, so no memory problem.
\makeatother

\usepackage{pgfplots}
\pgfplotsset{compat=newest}
\usepgfplotslibrary{fillbetween}
\usetikzlibrary{patterns}

\usepackage{siunitx}
\DeclareSIUnit{\calorie}{cal}
\usepackage[outline]{contour}

\usepackage{makecell}

\input{include}

\begin{document}


\title{Alchemical and Endpoint Free Energy Calculations at Scale}


\author{Shantenu Jha$^{1}$$^{,2}$, David W. Wright$^{3}$, Jumana Dakka$^{1}$,  Kristof Farkas-Pall$^{3}$\\
   \small{\emph{$^{1}$ Rutgers, the State University of New Jersey, Piscataway, NJ 08854, USA}}\\
   \small{\emph{$^{2}$ Brookhaven National Laboratory, Upton, New York, 11973}}\\
   \small{\emph{$^{3}$ University College London, London, UK, WC1H 0AJ}}
}


\date{}
\maketitle

\begin{abstract} 

\end{abstract}


% ---------------------------------------------------------------------------
% INTRODUCTION 1
% ---------------------------------------------------------------------------
\section{Introduction}\label{sec:intro}

The efficacy of drug treatments depends on how tightly small molecules bind to their target proteins. Quantifying the strength of these interactions (the so called ‘binding affinity’) is a grand challenge of computational chemistry, the surmounting of which could revolutionize drug design and provide the platform for patient specific medicine. Recently, improvements in computational power and algorithm design mean that reliably quantifying binding affinities from molecular simulation is now becoming a genuine possibility. Exploiting these advances and further refining the technologies involved requires the marshaling of huge simulation campaigns, and impacting clinical or industrial decision making means that computations must be turned around in timescales of hours or days. 


% ---------------------------------------------------------------------------
% Scientific Motivation
% ---------------------------------------------------------------------------

\section{Scientific Motivation}\label{sec:motivation}


% ---------------------------------------------------------------------------
% Binding Affinity Calculation Protocols
% ---------------------------------------------------------------------------

\section{Binding Affinity Calculation Protocols}\label{sec:bac}

\subsection{Alchemical Protocol (TIES)}\label{sec:ties}

\subsection{Endpoint Protocol (ESMACS)}\label{sec:esmacs}

\section{Computational Challenges}\label{sec:cc}

\section{Solution}\label{sec:solution}

\section{Impact of Solution}\label{sec:impact}

\section{Analysis of Solution}\label{sec:analysis}

% ---------------------------------------------------------------------------
% Demonstration
% ---------------------------------------------------------------------------
\section{Demonstration}\label{sec:demo}


% ---------------------------------------------------------------------------
% BIBLIOGRAPHY
% ---------------------------------------------------------------------------
\bibliographystyle{abbrv}
\bibliography{}

\end{document}

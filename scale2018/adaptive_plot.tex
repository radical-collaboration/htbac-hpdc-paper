\begin{figure}[h!]
  \centering
  \begin{tikzpicture}
  \begin{axis} [scale=0.3, xmin=0, xmax=1, yticklabels={,,}, xtick={0, 1}]
  \addplot+[name path=true, mark=none, smooth, color=black] table {dg-tot-scaled.dat};

  \addplot[name path=approx_1, mark=o] coordinates {(0.00, 0.37126) (0.50, -51.44248) (1.00, -51.30041)};

  \addplot fill between[
    of = true and approx_1,
    split,
    every even segment/.style = {pattern color=red!50, pattern=north west lines},
    every odd segment/.style = {pattern color=red!50, pattern=north west lines},
    soft clip={domain=0:1},
  ];

  \end{axis}
  \end{tikzpicture}%
  \begin{tikzpicture}
  \begin{axis} [scale=0.3, xmin=0, xmax=1, yticklabels={,,}, xtick={0, 1}]
  \addplot+[name path=true, mark=none, smooth, color=black] table {dg-tot-scaled.dat};

  \addplot[name path=approx_1, mark=o] coordinates {
  (0.00, 0.37126)
  (0.20, -1.01429)
  (0.50, -51.44248)
  (0.80, -53.15965)
  (1.00, -51.30041)};

  \addplot fill between[
    of = true and approx_1,
    split,
    every even segment/.style = {pattern color=red!50, pattern=north west lines},
    every odd segment/.style = {pattern color=red!50, pattern=north west lines},
    soft clip={domain=0:1},
  ];

  \end{axis}
  \end{tikzpicture}%
  \begin{tikzpicture}
  \begin{axis} [scale=0.3, xmin=0, xmax=1, yticklabels={,,}, xtick={0, 1}]
  \addplot+[name path=true, mark=none, smooth, color=black] table {dg-tot-scaled.dat};

  \addplot[name path=approx_1, mark=o] coordinates {
  (0.00, 0.37126)
  (0.20, -1.01429)
  (0.30, 1.977340)
  (0.40, 4.27343)
  (0.50, -51.44248)
  (0.60, -55.19837)
  (0.80, -53.15965)
  (1.00, -51.30041)};

  \addplot fill between[
    of = true and approx_1,
    split,
    every even segment/.style = {pattern color=red!50, pattern=north west lines},
    every odd segment/.style = {pattern color=red!50, pattern=north west lines},
    soft clip={domain=0:1},
  ];

  \end{axis}
  \end{tikzpicture}

  \caption{Adaptive quadrature of the function $f(\lambda) = \partial U(\lambda)/\partial \lambda$ in the interval $[0, 1]$ using the trapezoidal rule. The figures from left to right show successive levels of recursion or bisection of the interval. The interpolation error is the difference (shaded area) between the interpolant and the actual function. If the error in an interval is too large, the interval is bisected.}
  \label{fig:adaptive}
\end{figure}
